%\makeglossary
\makenoidxglossaries
%**************************************************************
% Glossary definition
%**************************************************************
\newglossaryentry{package} {
	name=package,
	description={ in informatica è, un raggruppamento di classi, metodi programmi, librerie e procedure che sono logicamente collegate tra di loro
	},
	plural=packages
}
\newglossaryentry{task} {
	name=task,
	description={ è un compito secondo la definizione dello standard ISO/IEC 12207
	},
	plural=tasks
}
\newglossaryentry{Gantt} {
	name=Gantt,
	description={  è un diagramma di supporto alla gestione dei progetti
	}
}
\newglossaryentry{custom} {
	name=custom,
	description={ è un attributo con cui si indica un manufatto, un dispositivo, o un componente, progettato e realizzato su misura in base alle necessità dell'acquirente o della funzione specifica che è destinato ad assolvere
	}
}
\newglossaryentry{API} {
	name=API,
	description={ è l'acronimo per Application Programming Interface (in italiano interfaccia di programmazione di un'applicazione), in informatica, indica ogni insieme di procedure disponibili al programmatore, di solito raggruppate a formare un set di strumenti specifici per l'espletamento di un determinato compito all'interno di un certo programma. Spesso con tale termine si intendono le librerie software disponibili in un certo linguaggio di programmazione
	}
}
\newglossaryentry{provider} {
	name=provider,
	description={ è un'espressione inglese che indica imprese che forniscono servizi di vario tipo,  italiano si è soliti tradurre "service provider" letteralmente, chiamando l'impresa "fornitore di servizi"
	},
	plural=providers
}
\newglossaryentry{token} {
	name=token,
	description={ con questo termine si fa riferimento ai Json Web Token (o JWT) che sono stringhe di caratteri alfanumerici che tipicamente incapsulano le credenziali di sicurezza per una sessione di login, identificando l'utente della stessa. Questi token sono segnati da una chiave del server grazie la quale è possibile verificarne la veridicità
	}
}
\newglossaryentry{header} {
	name=header,
	description={ o intestazione, è una parte del pacchetto ,inviato nelle richieste http, che contiene le informazioni di controllo necessarie al funzionamento della rete cioè le informazioni di protocollo aggiunte di strato in strato
	},
	plural=headers
}
\newglossaryentry{endpoint} {
	name=endpoint,
	description={ è l'url attraverso il quale un'applicazione client può accedere ad un servizio web. Lo stesso servizio web di solito ha endpoint multipli
	},
	plural=endpoints
}
\newglossaryentry{JSON} {
	name=JSON,
	description={  acronimo di Javascript Object Notation, è un formato adatto all'interscambio di dati fra applicazioni client-server
	}
}
\newglossaryentry{REST-based} {
	name=REST-based,
	description={ fa riferimento ad architetture o chiamate  basate sul protocollo REST (vedi sez. \ref{rest})
	}
}
\newglossaryentry{JOIN} {
	name=JOIN,
	description={  è una clausola del linguaggio SQL che serve a combinare (unire) le tuple di due o più relazioni di un database tramite l'operazione di congiunzione (od unione) dell'algebra relazionale
	}
}
\newglossaryentry{SQL} {
	name=SQL,
	description={ è un linguaggio standardizzato per database basati sul modello relazionale
	}
}
\newglossaryentry{query} {
	name=query,
	description={ indica l'interrogazione da parte di un utente di un database, strutturato tipicamente secondo il modello relazionale, per compiere determinate operazioni sui dati
	},
	plural=queries
}
\newglossaryentry{framework} {
	name=framework,
	description={  architettura o struttura di supporto su cui un programma può essere creato. In genere è composto da una serie di librerie e strumenti di sviluppo
	},
	plural=frameworks
}
\newglossaryentry{virtual machine} {
	name=virtual machine,
	description={ bla vla %todo
	},
	plural=virtual machines
}
\newglossaryentry{Design Pattern} {
	name=Design Pattern,
	description={ nell'ambito dell'ingegneria del software, un design pattern, è un concetto che può essere definito "una soluzione progettuale generale ad un problema ricorrente". Si tratta di una descrizione o modello logico da applicare per la risoluzione di un problema che può presentarsi in diverse situazioni durante le fasi di progettazione e sviluppo del software, ancor prima della definizione dell'algoritmo risolutivo della parte computazionale
	},
	plural=Design Patterns
}
\newglossaryentry{stateless} {
	name=stateless,
	description={ è un protocollo di comunicazione che tratta ogni richiesta come una transazione indipendente, scollegata da qualsiasi precedente richiesta, rendendo la comunicazione composta da coppie indipendenti di richiesta e risposta. Un protocollo stateless inoltre non richiede che il server mantenga le informazioni della sessione per ogni partner di comunicazione per la durata di multiple richieste
	}
}
\newglossaryentry{caching} {
	name=caching,
	description={ è il processo di immagazzinamento dati in una cache, che è una memoria temporanea
	}
}
\newglossaryentry{UML} {
	name=UML,
	description={ è un linguaggio di modellazione e specifica basato sul paradigma object-oriented
	}
}
\newglossaryentry{route} {
	name=route,
	description={ ci si riferisce alla definizione di un URI e a come esso risponde ad una specifica richiesta HTTP di un client
	},
	plural=routes
}
\newglossaryentry{generic} {
	name=generic,
	description={ bla vla %todo
	},
	plural=generics
}
\newglossaryentry{accessor} {
	name=accessor,
	description={per il linguaggio di programmazione C\# un accessor di una proprietà (o campo dati di una classe) contiene il codice associato con l'operazione di \textit{getting} (lettura) o il \textit{setting} (scrittura) della proprietà stessa. La dichiarazione un un accessor può contenerne uno di tipo \textit{get}, uno di tipo \textit{set} o entrambi
	},
	plural=accessors
}
\newglossaryentry{getter} {
	name=getter,
	description={ bla vla %todo
	},
	plural=getters
}
\newglossaryentry{setter} {
	name=setter,
	description={ bla vla %todo
	},
	plural=setters
}
\newglossaryentry{hostato} {
	name=hostato,
	description={ bla vla %todo
	}
}
\newglossaryentry{url} {
	name=url,
	description={ bla vla %todo
	}
}
\newglossaryentry{Debugger} {
	name=Debugger,
	description={ è un programma/software specificatamente progettato per l'analisi e l'eliminazione dei bug (debugging), ovvero errori di programmazione interni al codice di altri programmi. E' spesso compreso all'interno di un ambiente integrato di sviluppo (IDE)
	},
	plural=Debuggers
}
\newglossaryentry{logging} {
	name=logging,
	description={ bla vla %todo
	}
}
\newglossaryentry{REST} {
	name=REST-REST,
	description={ bla vla %todo
	}
}
%**************************************************************
% Acronyms definitions
%**************************************************************
\newacronym{bff}{BFF}{Botnet Finder Framework}

\newacronym{json}{JSON}{JavaScript Object Notation}

\newacronym{csv}{CSV}{Comma Separated Values}

\newacronym{iscx}{ISCX}{The Information Security Centre of Excellence}

\newacronym{roc}{ROC}{Receiver Operating Characteristic}

\newacronym{auc}{AUC}{Area Under the Curve}

\newacronym{ppv}{PPV}{Positive Predictive Values}

\newacronym{tpr}{TPR}{True Positive Rate}

\newacronym{fpr}{FPR}{False Positive Rate}

\newacronym{hits}{HITS}{Hyperlink-Induced Topic Search}

\newacronym{dbscan}{DBSCAN}{Density-Based Spatial Alustering of Applications with Noise}

\newacronym{eps}{EPS}{Encapsulated PostScript}

\newacronym{irc}{IRC}{Internet Relay Chat}

\newacronym{p2p}{P2P}{Peer-To-Peer}

\newacronym{http}{HTTP}{Hypertext Transfer Protocol}

\newacronym{osn}{OSN}{Online Social Networks}

\newacronym{cnc}{C\&C}{Command and Control}

\newacronym{elisa}{ELISA}{Elusive Social Army}

\newacronym{cdn}{CDN}{Content Delivery Network}

