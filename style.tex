%**************************************************************
% Packages
%**************************************************************
\usepackage[utf8]{inputenc}
\usepackage[T1]{fontenc}
%\usepackage[latin1]{inputenc}
\usepackage[italian]{babel}
\usepackage{graphicx}
\usepackage{float}
\usepackage[toc, acronym]{glossaries}
\usepackage[nottoc]{tocbibind}
\usepackage{transparent}
\usepackage{hyperref}
\usepackage{color}
\usepackage{epstopdf}
\usepackage{subcaption}
\usepackage{wrapfig}
\usepackage{tikz}
%\usepackage{tikz-uml}
%\usepackage{fancyhdr}
\usepackage{rotating}
\usepackage{listings,lstautogobble}
\usepackage{glossaries}
%\usepackage[acronym]{glossaries}
\usepackage{enumitem}

%\usepackage{biblatex} 
\usepackage[backend=biber]{biblatex}
\addbibresource{thesis.bib}  
\usepackage{geometry}
\geometry{outer=2cm,inner=4.5cm}


\definecolor{bluekeywords}{rgb}{0,0,1}
\definecolor{greencomments}{rgb}{0,0.5,0}
\definecolor{redstrings}{rgb}{0.64,0.08,0.08}
\definecolor{xmlcomments}{rgb}{0.5,0.5,0.5}
\definecolor{types}{rgb}{0.17,0.57,0.68}



\colorlet{punct}{red!60!black}
\definecolor{background}{HTML}{EEEEEE}
\definecolor{delim}{RGB}{20,105,176}
\colorlet{numb}{magenta!60!black}


\lstdefinelanguage{json}{
	basicstyle=\normalfont\ttfamily,
	numbers=left,
	numberstyle=\scriptsize,
	stepnumber=1,
	numbersep=8pt,
	showstringspaces=false,
	breaklines=true,
	frame=lines,
	backgroundcolor=\color{background},
	literate=
	*{0}{{{\color{numb}0}}}{1}
	{1}{{{\color{numb}1}}}{1}
	{2}{{{\color{numb}2}}}{1}
	{3}{{{\color{numb}3}}}{1}
	{4}{{{\color{numb}4}}}{1}
	{5}{{{\color{numb}5}}}{1}
	{6}{{{\color{numb}6}}}{1}
	{7}{{{\color{numb}7}}}{1}
	{8}{{{\color{numb}8}}}{1}
	{9}{{{\color{numb}9}}}{1}
	{:}{{{\color{punct}{:}}}}{1}
	{,}{{{\color{punct}{,}}}}{1}
	{\{}{{{\color{delim}{\{}}}}{1}
	{\}}{{{\color{delim}{\}}}}}{1}
	{[}{{{\color{delim}{[}}}}{1}
	{]}{{{\color{delim}{]}}}}{1},
}


\lstset{language=[Sharp]C,
	%captionpos=b,
	%numbers=left, %Nummerierung
	%numberstyle=\tiny, % kleine Zeilennummern
	%frame=lines, % Oberhalb und unterhalb des Listings ist eine Linie
	aboveskip=3mm,
	belowskip=3mm,
	showspaces=false,
	showtabs=false,
	breaklines=true,
	showstringspaces=false,
	breakatwhitespace=true,
	escapeinside={(*@}{@*)},
	commentstyle=\color{greencomments},
	morekeywords={partial, var, value, get, set, Public, HttpResponseMessage, async, HttpStatusCode, private, string, List<T>,OrganizationData,Organization},
	keywordstyle=\color{bluekeywords},
	stringstyle=\color{redstrings},
	basicstyle={\ttfamily},
	columns=flexible,	
	breaklines=true,
	tabsize=4,
	autogobble=true
}

\usepackage{setspace}
%\linespread{0.5}

%\pagestyle{fancy}


\usetikzlibrary{calc,trees,positioning,arrows,chains,shapes.geometric,decorations.pathreplacing,decorations.pathmorphing,shapes,matrix,shapes.symbols,fit}



%**************************************************************
% Setup
%**************************************************************

% Line spaceing
\renewcommand{\baselinestretch}{1.5}

% Links color
\hypersetup {
	colorlinks=true,
	linkcolor=black,
	urlcolor=blue,
	citecolor=black
}

% Macros
%\newcommand{\bff}{\acrshort{bff}}

% Diagrams
\tikzset{
	>=stealth',
	punktchain/.style={
		rectangle, 
		rounded corners, 
		fill=yellow!20,
		draw=black, very thick,
		text width=10em, 
		minimum height=3em, 
		text centered, 
		on chain},
	line/.style={draw, thick, <-},
	element/.style={
		tape,
		top color=white,
		bottom color=blue!50!black!60!,
		minimum width=8em,
		draw=blue!40!black!90, very thick,
		text width=10em, 
		minimum height=3.5em, 
		text centered, 
		on chain},
	every join/.style={->, thick,shorten >=1pt},
	decoration={brace},
	tuborg/.style={decorate},
	tubnode/.style={midway, right=2pt},
}

%**************************************************************
% Cover
%**************************************************************
\newcommand{\myName}{Andrea \textsc{Venier}}
\newcommand{\myTitle}{Integrazione tra applicazioni web mediante microservizi RESTful}
\newcommand{\mySubTitle}{Ampliamento delle funzionalità\\ di un’applicazione di Project Management}
\newcommand{\myDegree}{Corso di Laurea in Informatica}
\newcommand{\myUni}{Università di Padova}
\newcommand{\myDepartment}{Dipartimento di Matematica}
\newcommand{\myProf}{Gilberto \textsc{Filè}}
\newcommand{\myLocation}{Padova}
\newcommand{\myAA}{2015-2016}
\newcommand{\myTime}{Dicembre 2016}
\newcommand{\myCompany}{Athesys s.r.l}
\newcommand{\glo}[1]{\gls{#1}{\ped{G}}}
\newcommand{\glopl}[1]{\glspl{#1}{\ped{G}}}
%\newcommand{\glo}[1]{#1{\ped{G}}}

