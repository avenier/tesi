\documentclass[12pt,a4paper,twoside,openany,english]{book}
%**************************************************************
% Packages
%**************************************************************
\usepackage[utf8]{inputenc}
\usepackage[T1]{fontenc}
%\usepackage[latin1]{inputenc}
\usepackage[italian]{babel}
\usepackage{graphicx}
\usepackage{float}
\usepackage[toc, acronym]{glossaries}
\usepackage[nottoc]{tocbibind}
\usepackage{transparent}
\usepackage{hyperref}
\usepackage{color}
\usepackage{epstopdf}
\usepackage{subcaption}
\usepackage{wrapfig}
\usepackage{tikz}
%\usepackage{tikz-uml}
%\usepackage{fancyhdr}
\usepackage{rotating}
\usepackage{listings,lstautogobble}
\usepackage{glossaries}
%\usepackage[acronym]{glossaries}
\usepackage{enumitem}

%\usepackage{biblatex} 
\usepackage[backend=biber]{biblatex}
\addbibresource{thesis.bib}  
\usepackage{geometry}
\geometry{outer=2cm,inner=4.5cm}


\definecolor{bluekeywords}{rgb}{0,0,1}
\definecolor{greencomments}{rgb}{0,0.5,0}
\definecolor{redstrings}{rgb}{0.64,0.08,0.08}
\definecolor{xmlcomments}{rgb}{0.5,0.5,0.5}
\definecolor{types}{rgb}{0.17,0.57,0.68}



\colorlet{punct}{red!60!black}
\definecolor{background}{HTML}{EEEEEE}
\definecolor{delim}{RGB}{20,105,176}
\colorlet{numb}{magenta!60!black}


\lstdefinelanguage{json}{
	basicstyle=\normalfont\ttfamily,
	numbers=left,
	numberstyle=\scriptsize,
	stepnumber=1,
	numbersep=8pt,
	showstringspaces=false,
	breaklines=true,
	frame=lines,
	backgroundcolor=\color{background},
	literate=
	*{0}{{{\color{numb}0}}}{1}
	{1}{{{\color{numb}1}}}{1}
	{2}{{{\color{numb}2}}}{1}
	{3}{{{\color{numb}3}}}{1}
	{4}{{{\color{numb}4}}}{1}
	{5}{{{\color{numb}5}}}{1}
	{6}{{{\color{numb}6}}}{1}
	{7}{{{\color{numb}7}}}{1}
	{8}{{{\color{numb}8}}}{1}
	{9}{{{\color{numb}9}}}{1}
	{:}{{{\color{punct}{:}}}}{1}
	{,}{{{\color{punct}{,}}}}{1}
	{\{}{{{\color{delim}{\{}}}}{1}
	{\}}{{{\color{delim}{\}}}}}{1}
	{[}{{{\color{delim}{[}}}}{1}
	{]}{{{\color{delim}{]}}}}{1},
}


\lstset{language=[Sharp]C,
	%captionpos=b,
	%numbers=left, %Nummerierung
	%numberstyle=\tiny, % kleine Zeilennummern
	%frame=lines, % Oberhalb und unterhalb des Listings ist eine Linie
	aboveskip=3mm,
	belowskip=3mm,
	showspaces=false,
	showtabs=false,
	breaklines=true,
	showstringspaces=false,
	breakatwhitespace=true,
	escapeinside={(*@}{@*)},
	commentstyle=\color{greencomments},
	morekeywords={partial, var, value, get, set, Public, HttpResponseMessage, async, HttpStatusCode, private, string, List<T>,OrganizationData,Organization},
	keywordstyle=\color{bluekeywords},
	stringstyle=\color{redstrings},
	basicstyle={\ttfamily},
	columns=flexible,	
	breaklines=true,
	tabsize=4,
	autogobble=true
}

\usepackage{setspace}
%\linespread{0.5}

%\pagestyle{fancy}


\usetikzlibrary{calc,trees,positioning,arrows,chains,shapes.geometric,decorations.pathreplacing,decorations.pathmorphing,shapes,matrix,shapes.symbols,fit}



%**************************************************************
% Setup
%**************************************************************

% Line spaceing
\renewcommand{\baselinestretch}{1.5}

% Links color
\hypersetup {
	colorlinks=true,
	linkcolor=black,
	urlcolor=blue,
	citecolor=black
}

% Macros
%\newcommand{\bff}{\acrshort{bff}}

% Diagrams
\tikzset{
	>=stealth',
	punktchain/.style={
		rectangle, 
		rounded corners, 
		fill=yellow!20,
		draw=black, very thick,
		text width=10em, 
		minimum height=3em, 
		text centered, 
		on chain},
	line/.style={draw, thick, <-},
	element/.style={
		tape,
		top color=white,
		bottom color=blue!50!black!60!,
		minimum width=8em,
		draw=blue!40!black!90, very thick,
		text width=10em, 
		minimum height=3.5em, 
		text centered, 
		on chain},
	every join/.style={->, thick,shorten >=1pt},
	decoration={brace},
	tuborg/.style={decorate},
	tubnode/.style={midway, right=2pt},
}

%**************************************************************
% Cover
%**************************************************************
\newcommand{\myName}{Andrea \textsc{Venier}}
\newcommand{\myTitle}{Integrazione tra applicazioni web mediante microservizi RESTful}
\newcommand{\mySubTitle}{Ampliamento delle funzionalità\\ di un’applicazione di Project Management}
\newcommand{\myDegree}{Corso di Laurea in Informatica}
\newcommand{\myUni}{Università di Padova}
\newcommand{\myDepartment}{Dipartimento di Matematica}
\newcommand{\myProf}{Gilberto \textsc{Filè}}
\newcommand{\myLocation}{Padova}
\newcommand{\myAA}{2015-2016}
\newcommand{\myTime}{Dicembre 2016}
\newcommand{\myCompany}{Athesys s.r.l}
\newcommand{\glo}[1]{\gls{#1}{\ped{G}}}
\newcommand{\glopl}[1]{\glspl{#1}{\ped{G}}}
%\newcommand{\glo}[1]{#1{\ped{G}}}


\makeglossary
%\makenoidxglossaries
%**************************************************************
% Glossary definition
%**************************************************************
\newglossaryentry{package} {
	name=package,
	description={ in informatica è, un raggruppamento di classi, metodi programmi, librerie e procedure che sono logicamente collegate tra di loro
	},
	plural=packages
}
\newglossaryentry{task} {
	name=task,
	description={ è un compito secondo la definizione dello standard ISO/IEC 12207
	},
	plural=tasks
}
\newglossaryentry{Gantt} {
	name=Gantt,
	description={  è un diagramma di supporto alla gestione dei progetti
	}
}
\newglossaryentry{custom} {
	name=custom,
	description={ è un attributo con cui si indica un manufatto, un dispositivo, o un componente, progettato e realizzato su misura in base alle necessità dell'acquirente o della funzione specifica che è destinato ad assolvere
	}
}
\newglossaryentry{APIg} {
	name={API},
	description={ è l'acronimo per Application Programming Interface (in italiano interfaccia di programmazione di un'applicazione), in informatica, indica ogni insieme di procedure disponibili al programmatore, di solito raggruppate a formare un set di strumenti specifici per l'espletamento di un determinato compito all'interno di un certo programma. Spesso con tale termine si intendono le librerie software disponibili in un certo linguaggio di programmazione
	}
}

\newglossaryentry{provider} {
	name=provider,
	description={ è un'espressione inglese che indica imprese che forniscono servizi di vario tipo,  italiano si è soliti tradurre "service provider" letteralmente, chiamando l'impresa "fornitore di servizi"
	},
	plural=providers
}
\newglossaryentry{token} {
	name=token,
	description={ con questo termine si fa riferimento ai Json Web Token (o JWT) che sono stringhe di caratteri alfanumerici che tipicamente incapsulano le credenziali di sicurezza per una sessione di login, identificando l'utente della stessa. Questi token sono segnati da una chiave del server grazie la quale è possibile verificarne la veridicità
	}
}
\newglossaryentry{header} {
	name=header,
	description={ o intestazione, è una parte del pacchetto ,inviato nelle richieste http, che contiene le informazioni di controllo necessarie al funzionamento della rete cioè le informazioni di protocollo aggiunte di strato in strato
	},
	plural=headers
}
\newglossaryentry{endpoint} {
	name=endpoint,
	description={ è l'url attraverso il quale un'applicazione client può accedere ad un servizio web. Lo stesso servizio web di solito ha endpoint multipli
	},
	plural=endpoints
}
\newglossaryentry{JSONg} {
	name={JSON},
	description={  acronimo di Javascript Object Notation, è un formato adatto all'interscambio di dati fra applicazioni client-server
	}
}

\newglossaryentry{REST-based} {
	name=REST-based,
	description={ fa riferimento ad architetture o chiamate  basate sul protocollo REST (vedi sez. \ref{rest})
	}
}
\newglossaryentry{JOIN} {
	name=JOIN,
	description={  è una clausola del linguaggio SQL che serve a combinare (unire) le tuple di due o più relazioni di un database tramite l'operazione di congiunzione (od unione) dell'algebra relazionale
	}
}
\newglossaryentry{SQL} {
	name=SQL,
	description={ è un linguaggio standardizzato per database basati sul modello relazionale
	}
}
\newglossaryentry{query} {
	name=query,
	description={ indica l'interrogazione da parte di un utente di un database, strutturato tipicamente secondo il modello relazionale, per compiere determinate operazioni sui dati
	},
	plural=queries
}
\newglossaryentry{framework} {
	name=framework,
	description={  architettura o struttura di supporto su cui un programma può essere creato. In genere è composto da una serie di librerie e strumenti di sviluppo
	},
	plural=frameworks
}
\newglossaryentry{virtual machine} {
	name=virtual machine,
	description={ o macchina virtuale (VM), viene indicato un software che, attraverso un processo di virtualizzazione, crea un ambiente virtuale che emula tipicamente il comportamento di una macchina fisica grazie all'assegnazione di risorse hardware
	},
	plural=virtual machines
}
\newglossaryentry{Design Pattern} {
	name=Design Pattern,
	description={ nell'ambito dell'ingegneria del software, un design pattern, è un concetto che può essere definito "una soluzione progettuale generale ad un problema ricorrente". Si tratta di una descrizione o modello logico da applicare per la risoluzione di un problema che può presentarsi in diverse situazioni durante le fasi di progettazione e sviluppo del software, ancor prima della definizione dell'algoritmo risolutivo della parte computazionale
	},
	plural=Design Patterns
}
\newglossaryentry{stateless} {
	name=stateless,
	description={ è un protocollo di comunicazione che tratta ogni richiesta come una transazione indipendente, scollegata da qualsiasi precedente richiesta, rendendo la comunicazione composta da coppie indipendenti di richiesta e risposta. Un protocollo stateless inoltre non richiede che il server mantenga le informazioni della sessione per ogni partner di comunicazione per la durata di multiple richieste
	}
}
\newglossaryentry{caching} {
	name=caching,
	description={ è il processo di immagazzinamento dati in una cache, che è una memoria temporanea
	}
}
\newglossaryentry{UML} {
	name=UML,
	description={ è un linguaggio di modellazione e specifica basato sul paradigma object-oriented
	}
}
\newglossaryentry{route} {
	name=route,
	description={ ci si riferisce alla definizione di un URI e a come esso risponde ad una specifica richiesta HTTP di un client
	},
	plural=routes
}
\newglossaryentry{generic} {
	name=generic,
	description={ è un tipo di dato utilizzato negli algoritmi di programmazione che non ha la necessità di essere istanziato immediatamente, bensì può essere specificato in un secondo momento
	},
	plural=generics
}
\newglossaryentry{accessor} {
	name=accessor,
	description={per il linguaggio di programmazione C\# un accessor di una proprietà (o campo dati di una classe) contiene il codice associato con l'operazione di \textit{getting} (lettura) o il \textit{setting} (scrittura) della proprietà stessa. La dichiarazione un un accessor può contenerne uno di tipo \textit{get}, uno di tipo \textit{set} o entrambi
	},
	plural=accessors
}
\newglossaryentry{getter} {
	name=getter,
	description={ è un metodo di una classe utilizzato per leggere il valore di un campo dati della stessa
	},
	plural=getters
}
\newglossaryentry{setter} {
	name=setter,
	description={ è un metodo di una classe utilizzato per scrivere il valore di un campo dati della stessa
	},
	plural=setters
}

\newglossaryentry{URLg} {
	name={URL},
	description={ o Uniform Resource Locator è una sequenza di caratteri che identifica univocamente l'indirizzo di una risorsa in Internet, tipicamente presente su un host server
	}
}

\newglossaryentry{DBMSg} {
	name={DBMS},
	description={ sta per Database Management System o Sistema di gestione di basi di dati. Questo è un sistema software progettato per consentire la creazione, la manipolazione e l'interrogazione efficiente (da parte di uno o più utenti client) di database
	}
}
\newglossaryentry{Debugger} {
	name=Debugger,
	description={ è un programma/software specificatamente progettato per l'analisi e l'eliminazione dei bug (debugging), ovvero errori di programmazione interni al codice di altri programmi. E' spesso compreso all'interno di un ambiente integrato di sviluppo (IDE)
	},
	plural=Debuggers
}
\newglossaryentry{logging} {
	name=logging,
	description={ è la registrazione sequenziale e cronologica delle operazioni effettuate, da un utente, un amministratore o automatizzate, man mano che vengono eseguite dal sistema o applicazione
	}
}

\newglossaryentry{asset} {
	name=asset,
	description={ è una risorsa di un azienda
	}
}


\newglossaryentry{URIg} {
	name={URI},
	description={ o Uniform Resource Identifier, in informatica, si riferisce a una stringa che identifica univocamente una risorsa generica che può essere un indirizzo Web, un documento, un'immagine, un file, un servizio, un indirizzo di posta elettronica, ecc
	}
}

\newglossaryentry{cloud-based} {
	name=cloud-based,
	description={ si indica un paradigma basato, sul Cloud, di erogazione di risorse informatiche, come l'archiviazione, l'elaborazione o la trasmissione di dati, caratterizzato dalla disponibilità on-demand attraverso Internet a partire da un insieme di risorse preesistenti e configurabili
	}
}

%**************************************************************
% Acronyms definitions
%**************************************************************
\newacronym{CRM}{CRM}{Customer relationship management}

%\newacronym{JSON}{JSON}{JavaScript Object Notation}

\newacronym{http}{HTTP}{Hypertext Transfer Protocol}

\newacronym{DAO}{DAO}{Data Access Object}

\newacronym{DTO}{DTO}{Data Transfer Object}

%\newacronym{URI}{URI}{Uniform Resource Identifier}





%%% define the acronym and use the see= option
\newglossaryentry{URI}{type=\acronymtype, name={URI}, description={Uniform Resource Identifier}, first={Uniform Resource Identifier (URI)\glsadd{URIg}}, see=[Glossary:]{URIg}}

\newglossaryentry{DBMS}{type=\acronymtype, name={DBMS}, description={Database Management System}, first={Database Management System (DBMS)\glsadd{DBMSg}}, see=[Glossary:]{DBMSg}}

\newglossaryentry{JSON}{type=\acronymtype, name={JSON}, description={JavaScript Object Notation}, first={JavaScript Object Notation (JSON)\glsadd{JSONg}}, see=[Glossary:]{JSONg}}

\newglossaryentry{API}{type=\acronymtype, name={API}, description={Application Programming Interface}, first={Application Programming Interface (API)\glsadd{APIg}}, see=[Glossary:]{APIg}}

\newglossaryentry{URL}{type=\acronymtype, name={URL}, description={Uniform Resource Locator}, first={Uniform Resource Locator (URL)\glsadd{URLg}}, see=[Glossary:]{URLg}}

\title{\myTitle}
\author{\myName}
\date{\today}

\begin{document}

%**************************************************************
% Cover
%**************************************************************
\frontmatter
\begin{titlepage}
	\begin{center}
		\vbox to40pt{\vbox to\textheight{\vfill {\transparent{0.2}\includegraphics[width=10cm]{logo-unipd.png}} \vfill}}
	\end{center}
	\begin{minipage}{.20\textwidth}
		\includegraphics[height=2cm]{logo-department.png}
	\end{minipage}
	\hspace{40pt}
	\begin{minipage}{.80\textwidth}
		\begin{center}
			\begin{LARGE}
				\textsc{\textbf{\myUni}}\\
			\end{LARGE}
			\line(1,0){220}\\
			\begin{Large}
				\textsc{\myDepartment}\\
			\end{Large}
		\end{center}
	\end{minipage}
	\begin{center}
		\vfill
		\large{\textsc{\myDegree}}\\			
		\LARGE{\textsc{\textbf{\myTitle}}}\\
		\large{\mySubTitle}\\
		\vfill
		\begin{minipage}{0.4\textwidth}
			\begin{flushleft} \large
				\emph{Autore}\\
				\myName
			\end{flushleft}
		\end{minipage}
		\hfill
		\begin{minipage}{0.4\textwidth}
			\begin{flushright} \large
				\emph{Relatore} \\
				\myProf
			\end{flushright}
		\end{minipage}
		\vfill
		\line(1, 0){338} \\
		\begin{normalsize}
			\textsc{Anno Accademico \myAA}
		\end{normalsize}
	\end{center}
\end{titlepage}

%**************************************************************
% Colophon
%**************************************************************
\thispagestyle{empty}
\hfill
\vfill
\noindent \myName\ \textit{\myTitle,} \myDegree, Copyright \textcopyright\ \myTime.

\cleardoublepage
%\begingroup\onehalfspacing

\chapter*{Note Iniziali}\label{notes}
Il presente documento descrive l'esperienza di stage, della durata di 320 ore, svolta presso l'azienda \textbf{Athesys s.r.l.} di Padova.\\

\textit{\textbf{Convenzioni utilizzate}} : tutti i termini marcati con una 'G' a
pedice vengono spiegati nell'appendice Glossario alla fine del documento.\\
%**************************************************************
% Keywords
%**************************************************************
\chapter*{Keywords}\label{keywords}
Di seguito vengono descritte le parole chiave per poter capire al meglio questa tesi di laurea:

\paragraph*{CRM}
sta per \textit{Customer relationship management} e rappresenta un approccio per gestire le interazioni di un'azienda con i suoi clienti, attuali e potenziali.\\
Attraverso questa metodologia si punta a raccogliere ed analizzare i dati relativi ai clienti (email, sito aziendale, numeri telefonici ed in particolar modo lo storico delle offerte e degli ordini effettuati) con il fine di incrementare le vendite grazie alla fidelizzazione dello stesso mediante offerte e sconti mirati.\\
%Seguendo questi concetti sono nati, e si sono affermati, diversi software CRM e tra i maggiori esponenti possiamo trovare: \textbf{Salesforce} e \textbf{Microsoft Dynamics}.\\
%L'azienda \large{\myCompany{}}, basandosi sull'importanza e sulla diffusione dei suddetti software ha deciso di integrarli nella propria applicazione web di \textit{Project Management}.
 
\paragraph*{Project Management}
è la disciplina di che si occupa di sovraintendere la pianificazione, l'organizzazione e l'implementazione di un progetto.\\ Per progetto si intende un insieme ordinato di compiti da svolgere, caratterizzati da: vincoli di tempo (date di inizio e fine), risorse utilizzabili e risultati attesi.\\
Questa disciplina è fondamentale per controllare i costi, l'utilizzo delle risorse aziendali e assicurare che il prodotto finale dei relativi processi soddisfi le esigenze del cliente.\\
%In questo contesto \large{\myCompany{}} ha deciso di sviluppare \textbf{ADProject}, una soluzione software pensata ad-hoc per i propri clienti con l'obbiettivo di facilitare e automatizzare il suddetto insieme di attività.

%**************************************************************
% Abstract
%**************************************************************
\chapter*{Abstract}\label{abstract}
	%Il presente documento descrive l'esperienza di stage, della durata di 320 ore, svolta presso l'azienda \textbf{Athesys s.r.l.} di Padova.\\
	
	%\textit{Convenzioni utilizzate} : tutti i termini marcati con una 'G' a
	pedice vengono spiegati nell'appendice Glossario alla fine del documento.\\

	Segue una breve presentazione delle componenti coinvolte nella realizzazione del progetto di stage.
	
	\paragraph{ADProject}~\\
		L’azienda Athesys s.r.l. ha sviluppato un'applicazione web di \textit{Project Management}, chiamata ADProject, fornendo ai propri clienti la possibilità di avere un programma personalizzato in base alle loro esigenze.\\
		Questo software permette di:
		\begin{itemize}
			\itemsep-0.5em 
			\item censire le risorse aziendali (umane e materiali);
			\item creare progetti;
			\item suddividere il progetto in un insieme di attività da svolgere, pianificate al fine di portarlo a termine;
			\item suddividere le attività in \glo{task}, assegnando le risorse necessarie per il completamento degli stessi;
			\item tenere sotto controllo le scadenze, ritardi e costi di progetto;
			\item costruire diagrammi di \glo{Gantt};
			\item approvare e rendicontare le ore impiegate dal personale nello svolgimento dei \glo{task};
			\item importare i file di Microsoft Project, un importante software nell'ambito del Project Management.
			%TODO: incrementare le funzionalità
		\end{itemize} 
	\par
	
	\paragraph{SalesForce e Microsoft Dynamics}~\\
		La maggior parte delle aziende con un reparto commerciale ha la necessità di utilizzare software CRM per gestire al meglio le offerte, gli ordini e in generale i rapporti con i clienti. In questo ambito i due maggiori esponenti sono senza dubbio: \textit{SalesForce} e \textit{Microsoft Dynamics}, entrambe applicazioni \textit{cloud-based}.
	\par

	\paragraph{Elementi in comune tra Project Management e CRM}~\\
		I concetti di Project Management e CRM ad un primo sguardo possono sembrare non direttamente collegati, tuttavia sono anelli di una stessa catena ed operano su entità analoghe:
		\begin{itemize}
			\itemsep-0.5em 
			\item i prodotti da realizzare e vendere;
			\item la capacità di produzione, dipendente dal tempo e dalle risorse aziendali, che incide sulla disponibilità dei prodotti;	
			\item il cliente al quale devono essere imputati i costi delle risorse utilizzate e di sviluppo per la produzione di soluzioni software \glo{custom};
			\item alcune figure aziendali (tipicamente il personale del reparto contabilità) già censite all'interno del software CRM, annoverabili tra le risorse di un'azienda.
			%TODO:  da rivedere
		\end{itemize}
	\par
	
	\paragraph{Scopo del progetto e nascita di ADCrm}~\\
		%Ci sono moltissimi dati all'interno dei software CRM che possono essere utilizzati nell'ambito del Project Management, grazie ai quali verrebbero ridotti di molto i tempi di \textit{data-entry}, necessari per il corretto funzionamento di una nuova installazione di ADProject.\\
		L'esigenza ultima di Athesys s.r.l. è quella di fornire un prodotto completo e competitivo, attingendo a tutti i dati utili che i software CRM possono offrire, abbatendo i tempi di \textit{data-entry} e migliorando l'esperienza utente. Con questo fine, attraverso il progetto di stage, è stato realizzato \textbf{ADCrm}: un servizio web che si pone come ponte tra le due realtà, fornendo la possibilità ad ADProject di fruire dei dati presenti su SalesForce e Dynamics, rielaborandoli secondo le proprie esigenze, attraverso l'utilizzo di web \glo{API}.
	\par

%**************************************************************
% Table of contents, list of figures and list of tables
%**************************************************************
\tableofcontents
\listoffigures
\listoftables
\mainmatter

%**************************************************************
% Introduzione
%**************************************************************
\chapter{Il Progetto}\label{introduzione}
	\section{Piano di Lavoro}
		Il progetto, per quanto stabilito nel piano di lavoro dello stage, comporta l'integrazione di \textit{Salesforce} con \textit{ADProject}. Dato che la parte principale dell'applicazione è stata ultimata in anticipo rispetto al previsto, si è deciso di integrare nel servizio \textit{\textbf{ADCrm}} anche un secondo CRM: \textit{\textbf{Microsoft Dynamics}}.\\
		Nei successivi capitoli verrà descritta l'integrazione con entrambi i sistemi e verranno spiegate le sfide riscontrate in questo progetto, ponendo particolare attenzione alle parti riguardanti \textit{\textbf{SalesForce}}.
	\section{Studio delle applicazioni CRM}
		La prima parte dello stage consisteva nello studio di SalesForce per  acquisire dimestichezza con lo stesso, con i principi che stanno dietro ai software CRM ma soprattutto per identificare tutti i dati potenzialmente utili per l'integrazione con ADProject.\\
		In questa fase è stato analizzato il flusso di eventi necessari per il corretto utilizzo di Salesforce e le entità collegate.		
		\begin{enumerate}
			\item creazione di un'entità \textbf{Account} per un'azienda cliente o che potrebbe potenzialmente diventarlo, nella quale vengono censiti tutti i dati che si hanno a disposizione (nome, sito web, email, ricavi annuali, numero di dipendenti, area commerciale,...);
			\item creazione di una o più entità \textbf{Contatto} associate ad ogni \textbf{Account}, queste contengono tutti i dati relativi alle persone dell'azienda cliente, alle quali proporre offerte commerciali;
			\item inserimento nel CRM di ogni \textbf{Prodotto} che si intende vendere,associato ad un'apposita descrizione ed un prezzo di listino;
			\item creazione di un'entità \textbf{Offerta} alla quale associare una serie di prodotti (scontabili rispetto al prezzo di listino), nel momento in cui si vuole proporre un'offerta commerciale ad un cliente;
			\item creazione di un entità \textbf{Ordine}, contenente tutti i dati di spedizione, fatturazione etc, nel momento in cui l'offerta viene approvata dal cliente.
		\end{enumerate}
		I software CRM offrono un insieme di operazioni e funzionalità molto più vasto, ma non utili (al momento) ai fini dell'integrazione con ADProject.\\
		\paragraph{Le classi di dati}~\\
			Attraverso lo studio di SalesForce e di ADProject sono emerse le classi di dati che sarebbe utile importare dal CRM:
			\begin{itemize}
				\item \textbf{Account:} ogni progetto creato attraverso ADProject deve essere imputato ad un'azienda cliente
				\item \textbf{Contatti:} ogni Account deve avere un contatto principale come referente
				\item \textbf{Offerte:} ogni progetto deve avere una motivazione per essere creato, questa corrisponde all'offerta commerciale
				\item \textbf{Prodotti:} i prodotti corrispondono all'output principale dei progetti
				\item \textbf{Famiglie di Prodotti:} è comodo e utile accorpare i prodotti in macro-famiglie in quanto rende più facile la ricerca degli stessi 
				\item \textbf{Utenti:} il personale aziendale che utilizza il CRM (tipicamente il reparto commerciale) è bene che sia annoverato tra le risorse aziendali
			\end{itemize}
		\par
		\subsection{Procedure di Autenticazione}
			Per avere l'accesso ai dati del CRM, l'applicazione ADCrm deve potersi autenticare presso i server di SalesForce e Dynamics. Entrambe le applicazioni permettono l'autenticazione attraverso protocollo OAuth2.0.
			\paragraph{OAuth2.0}
			è un protocollo aperto che permette, in maniera semplice e standard, di autenticare e autorizzare applicazioni web, mobile e desktop ad accedere ai dati che si vogliono esporre, senza fornire direttamente le credenziali dell'utente.\\
			La scelta di questo protocollo è stata obbligata, dato che è l'unico metodo d'autenticazione esposto dai CRM, tuttavia è un'ottima opzione in quanto permette di rendere più sicuro il sistema, limitando la quantità di dati sensibili salvati all'interno dell'applicazione; inoltre sgrava il sistema dalle procedure di cambio password dato che sono a carico del \glo{provider} OAuth (SalesForce e Dynamics).
			%TODO: da ricontrollare

			\paragraph{Flusso di autenticazione}~\\
			In questo paragrafo viene descritto il flusso di autenticazione per Salesforce, necessario per ottenere il \glo{token} d'accesso.
			Questo \glo{token} è composto da una serie di caratteri alfanumerici, che è necessario allegare all'\glo{header} della richiesta http, per recuperare i dati contenuti nel CRM.
				\begin{figure}[!hb]
					\centering
					\includegraphics[width=0.7\linewidth]{images/webServerAuthFlow}
					\caption{Flusso di autenticazione per SalesForce}
					\label{fig:webserverauthflow}
				\end{figure}
			\begin{enumerate}
				\item L'applicazione client, che vuole accedere ai dati del CRM, deve effettuare una chiamata http con metodo POST (ed uno specifico set di parametri) all'appropriato \glo{endpoint} di SalesForce;
				\item SalesForce risponde inviando al chiamante, una pagina web sulla quale l'utente dovrà inserire le proprie credenziali d'accesso al sito; %L'utente interagisce direttamente con l'\glo{endpoint} di autorizzazione, in questo modo l'applicazione che effettua la richiesta d'accesso non necessita di conoscere le credenziali d'autenticazione;
				\item Una volta confermato che l'applicazione è autorizzata ad accedere al CRM, SalesForce invia al richiedente un codice alfanumerico in formato \glo{JSON};
				\item L'applicazione deve quindi estrarre il codice ricevuto nel punto 3, inserirlo in una richiesta http con metodo POST insieme ad un set specifico di parametri, ed inoltrate la richiesta all'\glo{endpoint} designato;
				\item Nel caso in cui la richiesta sia stata effettuata con successo, il server risponde con un file \glo{JSON}, contenente i \glo{token} necessari ad interrogare il CRM;
				\item L'applicazione client utilizza i \glo{token} d'accesso forniti per effettuare richieste al CRM. 
			\end{enumerate}
		\subsection{API e Protocolli di interrogazione dati}
			Entrambi i CRM permettono l'interrogazione dei propri dati attraverso l'implementazione di protocolli \textit{\glo{REST-based}}. Questi offrono anche la possibilità di mettere in relazione dati differenti, in maniera simile al comportamento della clausola \glo{JOIN} per le \glo{query} \glo{SQL}, anche se con limitazioni.\\ 
			La sfida che si è presentata in quest'ambito è stata quella di trovare il modo migliore di recuperare tutti i dati necessari limitando al massimo il numero di richieste http inviate ai CRM, in quanto ogni nuova richiesta effettuata aumenta il tempo di re-indirizzamento dei dati recuperati verso ADProject.
			~\\
			SalesForce e Microsoft Dynamics sfruttano due protocolli molto diversi tra di loro:
			%TODO: migliorare
			\paragraph{SOQL}
			o \textit{\textbf{S}alesforce \textbf{O}bject \textbf{Q}uery \textbf{L}anguage} permette di recuperare i dati della propria organizzazione da SalesForce. SOQL è basato su richieste http ad uno specifico \glo{endpoint}, al cui url viene accodata una \glo{query} simile a quelle formulate nel linguaggio \glo{SQL}. In caso di richiesta corretta, Il CRM risponde con i dati richiesti in formato \glo{JSON}.
			\paragraph{OData}
			E' un protocollo aperto (Open Data Protocol) creato da Microsoft, che definisce un insieme di best-practices per costruire e utilizzare API RESTful.\\
			Dynamics permette l'interrogazione dei propri dati attraverso semplici richieste http basate su questo protocollo, restituendo i dati richiesti in formato \glo{JSON}.
			
		\subsection{Simulazione query}
			Si è presto reso necessario verificare che il processo di autenticazione funzionasse correttamente e che fosse possibile recuperare tutti i dati voluti attraverso le \glo{query}, senza tuttavia cominciare la fase di codifica di ADCrm (dato che modifiche in corso d'opera all'architettura dell'applicazione, avrebbero causato un costo eccessivo in tempo e sforzi).
			Per risolvere questo problema si è deciso di simulare tutte le chiamate http, che il sistema avrebbe effettuato, tramite \textit{Postman}.
			\paragraph{Postman Rest Client} è un applicazione che permette di comporre richieste http in maniera semplice e veloce, risultando particolarmente utile per testare il funzionamento di \glo{API} proprie e altrui.
			\par
						
			\begin{figure}[!hb]
				\centering
				\includegraphics[width=\linewidth]{images/postman}
				\caption{Screenshot di una richiesta GET a SalesForce}
				\label{fig:postman}
			\end{figure}
			Una volta simulate tutte le richieste è stato possibile iniziare la fase di progettazione avendo ben chiari i passaggi fondamentali richiesti a ADCrm.
		
%**************************************************************
% Tecnologie e Strumenti
%**************************************************************
\chapter{Tecnologie e Strumenti}\label{tecnologie}
\section{Frameworks e linguaggi di programmazione}
In questa sezione vengono descritti brevemente i linguaggi di programmazione usati nella scrittura di \textbf{ADProject} e di \textbf{ADCrm}.
Non tutti i linguaggi che verranno presentati in seguito sono stati usati nell'ambito del progetto di stage (in quanto limitato alla realizzazione del servizio ADCrm), tuttavia la loro contestualizzazione all'interno dei vari moduli sviluppati dall'azienda aiuta a capire meglio il funzionamento degli stessi.

\paragraph{.NET Framework}
è un \glo{framework} software sviluppato da Microsoft. Include una vasta classe di librerie chiamata \textit{Framework Class Library} (FLC) che permette l'interoperabilità tra i linguaggi di programmazione del \glo{framework}. I programmi scritti per .NET vengono eseguiti in un ambiente software conosciuto come \textit{Common Language Runtime} (CLR), una \textit{\glo{virtual machine}}  che fornisce servizi come la gestione delle eccezioni e la gestione della memoria. FCL e CLR insieme costituiscono .NET.\\
Il software di project management \textbf{ADProject} e il modulo \textbf{ADCrm} sono sviluppati su questo \glo{framework}.
%TODO da incrementare questa parte o specificarla meglio

\paragraph{ASP.NET}
 è un \glo{framework} server-side, sviluppato da Microsoft, per la creazione di applicazioni web, che permette ai programmatori di costruire pagine dinamiche, applicazioni e web services.
 ASP.NET viene usato per lo sviluppo di tutte le pagine web dinamiche che fungono da interfacce grafiche per ADProject.

\paragraph{Visual Basic .NET}
è un linguaggio di programmazione orientato agli oggetti sviluppato su .NET.\\
Con questo linguaggio è stato sviluppato ADProject, e sono tuttora in sviluppo numerosi moduli di questo software. La motivazione principale che ha portato i developers assegnati al progetto a scegliere questo linguaggio è stata la loro familiarità con lo stesso. 

\paragraph{C\#}
è un linguaggio di programmazione orientato agli oggetti sviluppato su .NET.
La sintassi del linguaggio prende spunto da C++ , da Java
e da Visual Basic.
Questo linguaggio sta alla base di tutto lo sviluppo del microservice ADCrm, soggetto di questa tesi. La scelta si potrebbe pensare poco sensata in quanto ADCrm deve integrarsi con un applicazione web interamente scritta con Visual Basic .NET, tuttavia ci sono delle forti motivazioni che hanno fatto propendere il team di sviluppo ad adottare questo linguaggio:
\begin{itemize}
	\itemsep-0.5em
	\item avendo scelto di adottare una struttura basata su microservices per lo sviluppo di ADCrm, ed essendo quindi un modulo fisicamente slegato da ADProject, è possibile scegliere un linguaggio di programmazione differente da VB.NET, pur rimanendo all'interno dei linguaggi offerti dal \glo{framework};
	\item i programmatori assegnati allo sviluppo del progetto hanno maggiore familiarità con questo linguaggio;
	\item C\# ha una vastissima comunity e mediamente è più facile reperire documentazione e template rispetto a VB.NET;
	\item è più probabile trovare futuri developer C\#, in caso di incremento e manutenzione di ADCrm.	
\end{itemize}

\section{Ambiente di sviluppo}
\paragraph{Visual Studio}
 è un potente ambiente di sviluppo multipiattaforma fornito da Microsoft ed è stata la scelta naturale per lo sviluppo di ogni componente di ADProject ed ADCrm.
Visual Studio supporta, oltre ai linguaggi e \glo{framework} sopracitati: C, C++, F\#, Html e Javascript. 
%TODO da specificare cosa vuol dire multipiattaforma

\chapter{Analisi}\label{analisi}

%**************************************************************
% Casi d'uso
%**************************************************************
\section{Casi d'uso}\label{usecase}
%TODO: da aggiungere alla tabella dei casi d'uso le precondizioni da ricontrollare le immagini se sbordano e da controllare se aggiungere testo all'inizio di questa sezione
I casi d'uso sono creati secondo lo standard UML 2.4 e sono identificati dalla seguente notazione:
\begin{center}
	UC[Codice]: [Titolo]
\end{center}
dove il \textbf{codice} è un numero progressivo identificativo di ogni requisito, gerarchico nel  caso di sotto-casi d'uso tramite la notazione \textit{CodiceUCPadre.CodiceSottoUC}. Per ogni caso d'uso deve inoltre essere indicato:
\begin{itemize}
	\itemsep-0.5em
	\item \textbf{Titolo:} breve ma non ambiguo
	\item \textbf{Attori:} principali e secondari coinvolti
	\item \textbf{Descrizione:} una descrizione sintetica del caso d'uso
	%\item \textbf{Precondizione:} condizione necessaria affinché il caso d'uso possa avvenire
	\item \textbf{Flusso principale degli eventi:} descrizione, eventualmente per punti, del main scenario del Caso d'uso
	\item \textbf{Postcondizione:} condizione del sistema dopo che il caso d'uso è avvenuto
\end{itemize}
\begin{small}

	\hypertarget{UC1}{}
	\subsection{Caso d'uso UC1: Configurazione Aziendale}
	
	\begin{figure}[H]
		\centering
		\includegraphics[scale=0.55]{images/useCase/UC1}
		\caption{Use Case 1 - Configurazione aziendale}
		%\caption{}
		\label{fig:uc1}
	\end{figure}
	\begin{longtable}{ | p{2.7cm} | p{12cm} |}
		\hline \textbf{Attori} & Utente con privilegi di scrittura\\ 
		\hline \textbf{Precondizione} & L’utente deve essere autenticato, deve possedere i privilegi di configurazione aziendale e deve essere stato configurato il collegamento ad un \gls{CRM}\\
		\hline \textbf{Descrizione} & Attraverso l’interfaccia di configurazione aziendale, l’utente coi permessi corretti è in grado di visualizzare le impostazioni riguardanti la configurazione aziendale e modificarle\\ 
		\hline \textbf{Scenario Principale} & \begin{enumerate}
			\itemsep-0.5em 
			\item L’utente visualizza la lista delle famiglie di prodotti presenti nel listino del \gls{CRM}  (UC1.1);
			\item L’utente seleziona la tipologia di risorse con cui andare a mappare ciascuna famiglia di prodotti  (UC1.2);
			\item L’utente mappa eventuali famiglie di prodotti non mappate o prodotti specifici di un’offerta che non sono stati mappati  (UC1.3).
			
		\end{enumerate}
		\\ 
		\hline \textbf{Postcondizione} & Sono state visualizzate ed eventualmente modificate le associazioni tra le famiglie di prodotti e le tipologie di risorse\\ 
		\hline 
	\end{longtable}
	
	\hypertarget{UC1.2}{}
	\subsection{Caso d'uso UC1.2: Selezione della tipologie di risorsa con cui mappare ciascuna famiglia di prodotti}
	
	\begin{figure}[H]
		\centering
		\includegraphics[width=\linewidth]{images/useCase/UC1_2}
		\caption{Use Case 1.2 - Mappatura tipologie risorse / famiglie prodotti}
		%\caption{}
		\label{fig:uc1.2}
	\end{figure}

	\begin{longtable}{ | p{2.7cm} | p{12cm} |}
		\hline \textbf{Attori} & Utente con privilegi di scrittura\\ 
		\hline \textbf{Precondizione} & L’utente deve essere autenticato, deve possedere i privilegi di configurazione aziendale e deve essere stato configurato il collegamento ad un \gls{CRM}\\
		\hline \textbf{Descrizione} & Attraverso l’interfaccia di configurazione aziendale, l’utente coi permessi corretti è in grado di specificare con quale tipologia di risorsa (figura professionale, \glo{asset} o servizio) dovrà essere mappata ciascuna tipologia di prodotto proveniente dal \gls{CRM}\\ 
		\hline \textbf{Scenario Principale} & \begin{enumerate}
			\itemsep-0.5em 
			\item L’utente mappa una famiglia di prodotti con le figure professionali necessarie per un progetto  (UC1.2.1);
			\item L’utente mappa una famiglia di prodotti con gli asset necessarie per un progetto  (UC1.2.2);
			\item L’utente mappa una famiglia di servizi necessari per un progetto  (UC1.2.3);
			\item L’utente mappa una famiglia di prodotti come prodotti venduti (UC1.2.4).
			
		\end{enumerate}
		\\ 
		\hline \textbf{Estensioni} & \begin{enumerate}
			\item L’utente visualizza un messaggio di errore se tenta di cambiare la mappatura di una famiglia dopo che sono già stati importati prodotti relativi a quella famiglia  (UC1.4);
			
		\end{enumerate}
		\\ 
		\hline \textbf{Postcondizione} & Per almeno una famiglia di prodotto è stata modificata l’associazione ad una tipologia di risorsa. \\ 
		\hline 
	\end{longtable}
	
	\hypertarget{UC1.3}{}
	\subsection{Caso d'uso UC1.3: Mappatura famiglie di prodotti aggiunte successivamente con prodotti specifici di un offerta}
	\begin{figure}[H]
		\centering
		\includegraphics[scale=0.6]{images/useCase/UC1_3}
		\caption{Use Case 1.3 - Mappatura famiglie prodotti / prodotti offerta}
		%\caption{}
		\label{fig:uc1.3}
	\end{figure}
	\begin{longtable}{ | p{2.7cm} | p{12cm} |}
		\hline \textbf{Attori} & Utente con privilegi di scrittura\\
		\hline \textbf{Precondizione} &  L’utente deve essere autenticato, deve possedere i privilegi di configurazione di progetto, deve essere nella pagina di configurazione del progetto, deve aver selezionato il progetto e deve aver associato al progetto un’offerta che presenta prodotti con una mappatura anomala\\
		\hline \textbf{Descrizione} & Attraverso l’interfaccia di configurazione di progetto, l’utente coi permessi corretti è in grado di effettuare una mappatura \textit{just-in-time} di eventuali prodotti legati all’offerta che appartengono ad una famiglia di prodotti non mappata (perché ad esempio è stata aggiunta da poco nel \gls{CRM}) o che non sono associati ad una famiglia (perché magari sono prodotti associati puntualmente all’offerta\\ 
		\hline \textbf{Scenario Principale} & \begin{enumerate}
			\itemsep-0.5em 
			\item L’utente visualizza la lista dei prodotti non associati ad alcuna famiglia del \gls{CRM}  (UC1.3.1);
			\item L’utente seleziona la categoria di risorsa con cui mappare ciascun prodotto  (UC1.3.2);
			\item L’utente visualizza la lista delle famiglie di prodotti non mappate  (UC1.3.3);
			\item L’utente seleziona la categoria di risorsa con cui mappare ciascuna famiglia  (UC1.3.4).
			
		\end{enumerate}
		\\ 
		\hline \textbf{Postcondizione} & Sono state potenzialmente modificate le mappature dei prodotti\\ 
		\hline 
	\end{longtable}
	
	\hypertarget{UC2}{}
	\subsection{Caso d'uso UC2: Gestione delle anagrafiche aziendali}
	\begin{figure}[H]
		\centering
		\includegraphics[scale=0.50]{images/useCase/UC2}
		\caption{Use Case 2 - Gestione anagrafiche aziendali}
		%\caption{}
		\label{fig:uc2}
	\end{figure}
	\begin{longtable}{ | p{2.7cm} | p{12cm} |}
		\hline \textbf{Attori} & Utente con privilegi di lettura, Utente con privilegi di scrittura\\
		\hline \textbf{Precondizione} &  L’utente deve essere autenticato, deve possedere almeno i privilegi di visualizzazione di configurazione di progetto. Non è necessario che l’applicazione sia collegata ad un \gls{CRM}\\ 
		\hline \textbf{Descrizione} & Attraverso l’interfaccia di gestione delle anagrafiche aziendali, l’utente coi permessi corretti è in grado di visualizzare le informazioni relative alle schede di anagrafica aziendali e modificarle\\ 
		\hline \textbf{Scenario Principale} & \begin{enumerate}
			\itemsep-0.5em 
			\item L’utente crea un’anagrafica di un’azienda  (UC2.1);
			\item L’utente modifica l’anagrafica di un’azienda già censita  (UC2.3);
			\item L’utente gestisce i contatti relativi ad un’azienda  (UC2.4);
			\item L’utente visualizza la lista delle aziende censite nel sistema  (UC2.5);
			\item L’utente visualizza i dettagli relativi ad una azienda censita nel sistema  (UC2.6).
			
		\end{enumerate}
		\\ 
		\hline \textbf{Postcondizione} & Sono state visualizzate ed eventualmente modificate le anagrafiche aziendali\\ 
		\hline 
	\end{longtable}
	
	\hypertarget{UC2.4}{}
	\subsection{Caso d'uso UC2.4: Gestione dei contatti relativi ad un'azienda}
	\begin{figure}[H]
		\centering
		\includegraphics[scale=0.52]{images/useCase/UC2_4}
		\caption{Use Case 2.4 - Gestione contatti aziendali}
		%\caption{}
		\label{fig:uc2.4}
	\end{figure}
	\begin{longtable}{ | p{2.7cm} | p{12cm} |}
		\hline \textbf{Attori} & Utente con privilegi di scrittura\\
		\hline \textbf{Precondizione} &  L’utente deve essere autenticato, deve possedere almeno i privilegi di visualizzazione di configurazione di progetto. Non è necessario che l’applicazione sia collegata ad un \gls{CRM}\\  
		\hline \textbf{Descrizione} & Attraverso l’interfaccia di gestione delle anagrafiche aziendali, l’utente coi permessi corretti è in grado di visualizzare le informazioni sui contatti di riferimento per le diverse aziende ed eventualmente modificarle\\ 
		\hline \textbf{Scenario Principale} & \begin{enumerate}
			\itemsep-0.5em 
			\item L’utente crea un’anagrafica di contatto ;
			\item L’utente modifica l’anagrafica di un contatto già censito ;
			\item L’utente visualizza la lista dei contatti di riferimento per una specifica azienda ;
			\item L’utente visualizza i dettagli relativi ad un contatto.
			
		\end{enumerate}
		\\ 
		\hline \textbf{Postcondizione} & Sono state visualizzate ed eventualmente modificate le informazioni relative ai contatti di riferimento per un’azienda\\ 
		\hline 
	\end{longtable}
	
	\hypertarget{UC3}{}
	\subsection{Caso d'uso UC3: Gestione dei progetti}
	\begin{figure}[H]
		\centering
		\includegraphics[scale=0.55]{images/useCase/UC3}
		\caption{Use Case 3 - Gestione progetti}
		%\caption{}
		\label{fig:uc3}
	\end{figure}
	\begin{longtable}{ | p{2.7cm} | p{12cm} |}
		\hline \textbf{Attori} & Utente con privilegi di lettura, Utente con privilegi di scrittura\\
		\hline \textbf{Precondizione} &  L’utente deve essere autenticato, possedere almeno i privilegi di visualizzazione di configurazione di progetto, essere nella pagina di configurazione del progetto, aver selezionato il progetto, il tipo del progetto non deve essere Interno. È necessario che l’applicazione sia collegata ad un \gls{CRM} per quanto riguarda la parte relativa alle offerte\\  
		\hline \textbf{Descrizione} & Attraverso l’interfaccia di configurazione di progetto, l’utente coi permessi corretti è in grado di modificare le informazioni di progetto \\ 
		\hline \textbf{Scenario Principale} & \begin{enumerate}
			\itemsep-0.5em 
			\item L’utente associa un commerciale ad un progetto  (UC3.1);
			\item L’utente associa una delle aziende clienti censite al progetto  (UC3.2);
			\item L’utente associa una o più offerte libere al progetto selezionato  (UC3.3);
			\item L’utente gestisce il contatto principale di riferimento per un progetto  (UC3.4);
			\item L’utente visualizza la lista delle offerte libere (non associate ad alcun progetto) relative al cliente selezionato  (UC3.6);
			\item L’utente visualizza i dettagli delle offerte associate al progetto  (UC3.7);
			\item L’utente visualizza la lista di tutte le offerte libere  (UC3.8);
			\item L’utente può creare un progetto a partire da un’offerta  (UC3.5).
			
		\end{enumerate}
		\\ 
		\hline \textbf{Postcondizione} & Sono modificate le informazioni relative ad un progetto. \\ 
		\hline 
	\end{longtable}
	
	\hypertarget{UC3.4}{}
	\subsection{Caso d'uso UC3.4: Gestione del contatto principale del progetto}
		\begin{figure}[H]
		\centering
		\includegraphics[scale=0.5]{images/useCase/UC3_4}
		\caption{Use Case 3.4 - Gestione contatto principale}
		%\caption{}
		\label{fig:uc3.4}
	\end{figure}
	\begin{longtable}{ | p{2.7cm} | p{12cm} |}
		\hline \textbf{Attori} & Utente con privilegi di scrittura\\
		\hline \textbf{Precondizione} &  L'utente ha selezionato la voce per gestire il contatto principale del progetto\\ 
		\hline \textbf{Descrizione} & Attraverso l’interfaccia di configurazione di progetto, l’utente coi permessi corretti è in grado di modificare le informazioni relative al contatto principale di riferimento per un progetto (il PM lato cliente)\\ 
		\hline \textbf{Scenario Principale} & \begin{enumerate}
			\itemsep-0.5em 
			\item L’utente seleziona uno dei contatti associato al cliente selezionato per il progetto  (UC3.4.1);
			\item L’utente inserisce un nuovo contatto  (UC3.4.2);
			\item L’utente modifica i dettagli relativi al contatto di riferimento per il progetto  (UC3.4.3);
			\item L’utente visualizza i dettagli relativi al contatto inserito  (UC3.4.4).
			
		\end{enumerate}
		\\ 
		\hline \textbf{Estensioni} & \begin{enumerate}
			\item Viene visualizzato un messaggio d'errore se si tenta di aggiungere un contatto principale prima che sia associato un'azienda cliente al progetto (UC3.12);
			
		\end{enumerate}
		\\ 
		\hline \textbf{Postcondizione} & Sono state modificate le informazioni relative al contatto principale di riferimento per un progetto \\ 
		\hline 
	\end{longtable}
	
	\hypertarget{UC3.5}{}
	\subsection{Caso d'uso UC3.5: Creazione nuovo progetto da offerta}
		\begin{figure}[H]
		\centering
		\includegraphics[scale=0.55]{images/useCase/UC3_5}
		\caption{Use Case 3.5 - Creazione progetto da offerta}
		%\caption{}
		\label{fig:uc3.5}
	\end{figure}
	\begin{longtable}{ | p{2.7cm} | p{12cm} |}
		\hline \textbf{Attori} & Utente con privilegi di scrittura\\ 
		\hline \textbf{Precondizione} &  L'utente selezionata la voce per creare un nuovo progetto partendo da un offerta esistente\\ 
		\hline \textbf{Descrizione} & L'utente crea un progetto a partire dai dati di una specifica offerta, auto-completando quindi l'associazione tra progetto con account, offerta e contatto\\ 
		\hline \textbf{Scenario Principale} & \begin{enumerate}
			\itemsep-0.5em 
			\item L’utente inserisce i dati di base del progetto  (UC3.5.1);
			\item L’utente selezione il cliente su cui filtrare le offerte  (UC3.5.2);
			\item L’utente seleziona l’offerta da cui creare il nuovo progetto  (UC3.5.3).
			
		\end{enumerate}
		\\ 
		\hline \textbf{Postcondizione} & Viene creato un nuovo progetto\\ 
		\hline 
	\end{longtable}
\end{small}





%**************************************************************
% Requisiti
%**************************************************************
\section{Requisiti}\label{requisiti}
Ogni requisito dovrà essere classificato per tipo e importanza, utilizzando la seguente struttura:
\begin{center}
	R[importanza][tipo][codice]
\end{center}
\begin{itemize}
	\item \textbf{Importanza} può assumere i seguenti valori:
	\begin{description}
		\item[1:] requisito desiderabile
		\item[2:] requisito opzionale
		\item[3:] requisito obbligatorio
	\end{description}
	\item \textbf{Tipo} può assumere i seguenti valori:
	\begin{description}
		\item[F:] Funzionale, descrive i servizi o le funzioni offerte dal sistema
		\item[Q:] Di Qualità, descrive i requisiti sulla qualità offerte dal sistema
		\item[P:] Prestazionale, descrive i requisiti sulle prestazioni offerte dal sistema
		\item[V:] Vincolo, descrive i vincoli sui servizi offerti dal sistema
	\end{description}
	\item \textbf{Codice} è un numero progressivo univoco per ogni requisito, indipendente da importanza e tipo. Nel caso si abbia un sotto-requisito codice può anche essere espresso in modo gerarchico tramite la notazione:
	\begin{center}
		\textit{CodiceRequistoPadre.CodiceSottorequisito}
	\end{center}
\end{itemize}
%Ogni requisito deve essere correlato da una sintetica ma precisa descrizione. Per ogni requisito bisogna indicarne le fonti, che posso essere il capitolato o uno o più casi d'uso.
%TODO: da controllare tutto e da controllare se aggiungere testo all'inizio di questa sezione
\begin{small}
\begin{longtable}{|r l|p{2.5cm}|p{10cm}|}
	\hline
	\multicolumn{2}{|c|}{\textbf{Codice}} & \textbf{Tipologia} & \textbf{Descrizione}\tabularnewline
	\hline
	& \hypertarget{R-3F1}{R-3F1} & Funzionale
	
	Obbligatorio & E' necessario avere un account con permessi amministrativi per Salesforce\tabularnewline
	\hline
	 & R-3F1.1 & Funzionale
	
	Obbligatorio & E' necessario che Salesforce sia configurato per esporre le API necessarie a recuperare i dati desiderati\tabularnewline
	\hline
	& \hypertarget{R-3F2}{R-3F2} & Funzionale
	
	Obbligatorio & L'applicazione ADCrm deve poter interrogare Salesforce attraverso le API RESTful che vengono esposte
	\tabularnewline
	\hline
	& R-3F2.1 & Funzionale
	
	Obbligatorio & L'applicazione ADCrm deve poter interrogare Salesforce per otterere tutti i dati relativi alle Famiglie di Prodotti, recuperando: il campo identificativo della famiglia di prodotti, una descrizione della stessa ed un eventuale id del Parent\tabularnewline
	\hline
	 & R-3F2.2 & Funzionale
	
	Obbligatorio & L'applicazione ADCrm deve poter interrogare Salesforce per otterere tutti i dati relativi agli Account delle aziende clienti recuperando: il campo identificativo, il nome e tutti i dati informativi riguardanti l'azienda (email,telefono,sito web,città,...)\tabularnewline
	\hline
	& R-3F2.3 & Funzionale
	
	Obbligatorio & L'applicazione ADCrm deve poter interrogare Salesforce per otterere tutti i dati relativi ai contatti di un azienda recuperando: il campo identificativo, il nome, il cognome e tutti i dati informativi riguardanti il contatto (email, telefono, fax, ...)\tabularnewline
	\hline
	& R-3F2.4 & Funzionale
	
	Obbligatorio & L'applicazione ADCrm deve poter interrogare Salesforce per otterere tutti i dati relativi alle offerte commerciali legate ad un azienda recuperando: il campo identificativo, l'identificativo aziendale, il prezzo, lo sconto e se l'offerta ha ordini associati \tabularnewline
	\hline
	& R-3F2.5 & Funzionale
	
	Obbligatorio & L'applicazione ADCrm deve poter interrogare Salesforce per otterere tutti i dati relativi ai prodotti legati ad un offerta commerciale recuperando: il campo identificativo, il nome, la famiglia di prodotti a cui appartiene, l'identificativo dell'offerta, il prezzo, la quantità\tabularnewline
	\hline
	& R-3F2.6 & Funzionale
	
	Obbligatorio & L'applicazione ADCrm deve poter interrogare Salesforce per otterere tutti i dati relativi agli utenti dell'azienda che utilizzano il CRM recuperando: il campo identificativo, il nome, il cognome e la email\tabularnewline
	\hline
	& R-3F3 & Funzionale
	
	Obbligatorio & L'applicazione ADCrm deve esporre delle API REST per poter essere interrogata da ADProject con il fine di fornire i dati recuperati da Salesforce\tabularnewline
	\hline
	\caption{Tabella requisiti} 
	\tabularnewline
\end{longtable}

\end{small}
%**************************************************************
% Progettazione
%**************************************************************
\chapter{Progettazione}\label{progettazione}
In questo capitolo viene dettagliatamente descritta l'applicazione, presentando: 
\begin{enumerate}
	\itemsep-0.5em
	\item I \glo{Design Pattern} utilizzati, contestualizzando il loro utilizzo;
	\item La descrizione dell'architettura ad alto livello del sistema;
	\item Le \glo{API} REST esposte per permettere al client di comunicare con l'applicazione;
	\item Una descrizione più dettagliata delle classi che rappresentano il cuore dell'applicazione.
\end{enumerate}

\section{Design pattern}\label{design_pattern}
\subsection{Microservices}
Lo stile architetturale a microservizi è un approccio per sviluppare una singola applicazione come un insieme di piccoli servizi autonomi che interagiscono tra di loro, ognuno dei quali gestisce i suoi processi e comunica attraverso meccanismi snelli, spesso \glo{API} http.
Questa architettura permette di velocizzare i tempi di sviluppo (essendo i servizi autonomi tra di loro essi possono raggiungere l'ambiente di produzione anche in momenti separati ed inoltre possono essere sviluppati da persone diverse), aumentare la resilienza del sistema (se uno dei servizi smette di funzionare essendo separato dagli altri, non pregiudica il funzionamento del sistema) ed è più facilmente scalabile rispetto ad un architettura monolitica.\\
Si è deciso di sviluppare ADCrm seguendo quest'ottica, ottenendo in questo modo:
\begin{itemize}
	\itemsep-0.5em
	\item una netta separazione dei compiti delle varie componenti del sistema;
	\item la possibilità di collegarsi a nuovi CRM senza dover modificare in alcun modo ADProject, in quanto esso comunica con il servizio attraverso un interfaccia comune;
	\item un aumento  della resilienza e la scalabilità dell'applicazione.
\end{itemize}


\subsection{RESTful}
REST o \textit{\textbf{Re}presentational \textbf{S}tate \textbf{T}ransfer} è un architettura software che permette di rendere interoperabili sistemi di computer attraverso internet. 
Un servizio web REST permette ai sistemi richiedenti, di accedere e manipolare una rappresentazione testuale delle sue risorse web attraverso un insieme predefinito di operazioni \glo{stateless}.
Utilizzare i metodi definiti per il protocollo http (GET, POST, PUT, DELETE) è il modo più comune accedere alle risorse web del sistema che le espone.
Questa architettura attraverso l'utilizzo del protocollo \glo{stateless} e di insiemi di operazioni standard, mira a garantire ottime performance, essere affidabile e scalabile senza dover necessitare di modificare tutto il sistema nel caso di cambiamenti a qualche sua componente.
L'architettura REST viene utilizzata da \textbf{ADCrm\textbf{}} per recuperare ed esporre a sua volta le risorse presenti sul CRM.
Concretamente sarà presente quindi un microservizio per ogni CRM che si vorrà collegare all'applicazione.

\subsection{DTO \& DAO}
Il DTO (\textit{Data Transfer Object}) e DAO (\textit{Data Access Object}) sono due design pattern architetturali spesso usati in congiunzione.\\
La loro funzione principale è quella di disaccoppiare il \textit{data layer} (la parte in cui risiedono i dati) dalla parte di \textit{business logic} (ciò che rappresenta la logica applicativa) dell'applicazione, aumentando cosi la separazione tra le varie componenti e quindi la loro modularità e manutenibilità.\\
Le classi di oggetti DTO al loro interno avranno solamente campi dati accedibili attraverso metodi \textit{getter} e \textit{setter}, mentre gli oggetti DAO implementeranno i metodi, utilizzati dalla \textit{business logic}, per accedere ai dati dell'applicazione utilizzando quindi i sopracitati \textit{Data Transfer Object}.

\subsection{Factory Method} \label{factory}
Questo design pattern creazionale, nell'ambito della programmazione orientata agli oggetti, permette di affrontare il problema della creazione di oggetti senza specificarne a priori l'esatta classe di appartenenza.
Il pattern fornisce un interfaccia per la creazione di un oggetto, ma lascia alle classi che la implementano la decisione di quale oggetto istanziare.\\

\begin{figure}[H]
	\centering
	\includegraphics[width=\linewidth]{images/abstract_factory_structure}
	\caption{Schema UML d'esempio per il Factory design pattern}
	\label{fig:abstractfactorystructure}
\end{figure}

In questo modo l'applicazione ADCrm non solo non ha bisogno di sapere a priori se deve costruire oggetti per \textit{Salesforce} o altri CRM, ma ha anche la possibilità di aver collegarsi a più applicazioni CRM contemporaneamente.


\section{Architettura del sistema}
Lo scopo dell'applicazione \textbf{ADCrm} è quello di rispondere alle richieste provenienti da ADProject, recuperando in tempo reale i dati desiderati dal CRM ed elaborarli per renderli fruibili.\\
Per supplire a questa necessità, ADCrm presenta un'architettura client-server, assumendone entrambi i ruoli:
\paragraph{Server}
L'applicazione assume il ruolo di server nel momento in cui deve assolvere alle richieste provenienti da ADProject mediante le \glo{API} REST. Deve quindi rispondere a due esigenze principali:
\begin{itemize}
	\item esporre \glo{API} REST per poter rispondere, in formato \glo{JSON}, alle richieste \gls{http} ricevute;
	\item salvare i dati, non sensibili, necessari per il processo di autenticazione e collegamento con il CRM su un piccolo database interno.
\end{itemize} 

Si è deciso di mantenere il database sul servizio ADCrm per togliere l'onere ad ADProject di conoscere: non solo le procedure di autenticazione al CRM, ma anche la tipologia di CRM che si va ad interrogare (Salesforce, Dynamics, ...). In questo modo si riesce a mantenere una netta separazione dei ruoli tra le varie componenti. 

\paragraph{Client}
L'applicazione assume il ruolo di client nel momento in cui deve interrogare il server del CRM per recuperare i dati di cui ha bisogno ADProject, deve quindi:
\begin{itemize}
	\item interrogare il server remoto attraverso il set di \glo{API} esposto;
	\item organizzare i dati ricevuti in modo tale che siano fruibili da ADProject;
	\item restituire gli stessi in formato \glo{JSON}.
\end{itemize}

Nelle prime fasi della progettazione si è valutata l'ipotesi di utilizzare un database per fare \glo{caching} dei dati provenienti dal CRM, così da limitare le richieste \gls{http} necessarie e da velocizzare le risposte fornite, tuttavia la necessità di ADProject di utilizzare sempre dati aggiornati ha fatto abbandonare quest'opzione.\\

La figura \ref{fig:architetturasistema} illustra l'architettura ad alto livello del sistema; una descrizione più dettagliata delle varie componenti verrà esposta nelle sezioni successive.

\begin{figure}[H]
	\centering
	\includegraphics[width=\linewidth]{images/architettura_sistema}
	\caption{Architettura ad alto livello}
	\label{fig:architetturasistema}
\end{figure}

\section{API REST}\label{apiRest}
Di seguito si trova la definizione delle \glo{API} REST esposte dal servizio \textbf{ADCrm}.\\
I metodi \gls{http} con cui chiamare tutti i successivi \glo{URL} sono metodi \textbf{GET} ed i dati ricevuti nelle risposte sono forniti in formato \glo{JSON}.

	\begin{small}
		\begin{longtable}{ | l | p{8cm} | }
			\hline \textbf{\glo{URL}} & \textbf{Descrizione Risposta}\\
			\hline api/organization/\{organizationId\}/ & Risponde con i dati riguardante l'organizzazione avente \textit{id = organizationId}\\
			\hline
		\end{longtable}		
	\end{small}
Questo primo \glo{endpoint} necessita di ulteriori spiegazioni in quanto l'\glo{URL} specificato sopra è da anteporre a di tutti quelli che seguiranno \\(eg. "\textbf{api/organization/\{organizationId\}/accounts}" nel caso dell'\glo{URL} "/accounts").\\
Inoltre si noti che è necessario da parte di chi effettua chiamate verso questo \glo{URL} (e quindi anche verso tutti gli altri) conoscere il campo \textit{organizationId} per poter ottenere una risposta. Questa restrizione è stata posta per far si che il sistema sia più sicuro, rendendo disponibili dei dati riguardanti l'azienda fruitrice del servizio solo se si conosce il corretto identificativo.
%TODO: decidere se sistemare la parte qua sopra
\newpage
\begingroup\onehalfspacing
	\begin{small}
		\begin{longtable}{ | l | p{8cm} | }
			\hline \textbf{\glo{URL}} & \textbf{Descrizione Risposta}\\
			\hline /accounts & Risponde con la lista contenente i dati di tutti gli account\\
			\hline /accounts/\{\textit{accountId}\} & Risponde con i dati dell'account avente \textit{id = accountId}\\    
			\hline /accounts/\{\textit{accountId}\}/contacts & Risponde con la lista contenente i dati di tutti gli i contatti legati all'account avente \textit{id = accountId}\\
			\hline /accounts/\{\textit{accountId}\}/proposals & Risponde con la lista contenente i dati di tutte le offerte commerciali legate all'account avente \textit{id = accountId}\\
			\hline /contacts/\{\textit{contactId}\} & Risponde con i dati del contatto avente \textit{id = contactId}\\
			\hline /proposals & Risponde con la lista contenente i dati di tutte le proposte commerciali\\
			\hline /proposals/\{\textit{proposalId}\} & Risponde con i dati della proposta commerciale avente \textit{id = proposalId}\\    
			\hline /proposals/\{\textit{proposalId}\}/products & Risponde con una lista contenente tutti i prodotti legati all'offerta commerciale avente \textit{id =  proposalId}\\
			\hline /products/\{\textit{productId}\} & Risponde con i dati del prodotto avente \textit{id = productId}\\
			\hline /users & Risponde con una lista contenente i dati di tutti gli utenti del CRM\\
			\hline /users/\{\textit{userId}\} & Risponde con i dati del utente avente \textit{id = userId}\\    
			\hline /productCategories & Risponde con una lista contenente i dati di tutte le famiglie di prodotti\\
			\hline /productCategories/\{\textit{categoryId}\} & Risponde con i dati della famiglia di prodotti avente \textit{id = categoryId}\\		
			\hline 
		\end{longtable}		
	\end{small}
\endgroup
\section{Progettazione di dettaglio}\label{progettazioneDiDettaglio}
In questa sezione verranno descritti in maniera dettagliata tutti i \glo{package}, le classi principali dell'applicazione e i metodi più importanti delle stesse al fine mostrare al meglio il progetto di stage svolto.

%\begin{figure}[H]
%	\centering
%	\includegraphics[width=\linewidth]{images/modulesDiagram}
%	\caption{Diagramma dei moduli di ADCrm}
%	\label{fig:generalUMLDiagram}
%\end{figure}

\begin{figure}[H]
	\centering
	\includegraphics[width=\linewidth]{images/general2}
	\caption{Diagramma UML di ADCrm}
	\label{fig:modulesdiagram}
\end{figure}
\newpage
Nella figura \ref{fig:modulesdiagram} viene illustrato il diagramma \glo{UML} delle classi dell'applicazione.

\subsection{Athesys.ADCrm.API}
\begin{figure}[H]
	\centering
	\includegraphics[width=\linewidth]{images/modules/API}
	\caption{Diagramma UML del package Athesys.ADCrm.API}
	\label{fig:api}
\end{figure}
\paragraph{Descrizione:} 
Questo \glo{package} è utilizzato per esporre le \glo{API} REST all'applicazione web ADProject. In esso vengono definite le \glo{route} che è possibile chiamare, ad ognuna di esse viene associato un metodo di un particolare controller. 





\subsection{Athesys.ADCrm.API.AppStart}
\paragraph{Descrizione:} 
Questo \glo{package} contiene la classe che si occupa di definire le \glo{route} che verranno esposte dall'applicazione.
\vfill
\subsection{Athesys.ADCrm.API.Controllers}
\begin{figure}[H]
	\centering
	\includegraphics[width=\linewidth]{images/modules/Controllers}
	\caption{Diagramma UML del package Athesys.ADCrm.API::Controllers}
	\label{fig:controllers}
\end{figure}
\paragraph{Descrizione:} Questo \glo{package} contiene tutte le classi controller contenenti i metodi necessari a rispondere alle richieste \gls{http} effettuate alle \glo{route}.


\subsubsection{AccountsController (class)} \label{accountsController}
\paragraph{Descrizione:}
Classe che serve per gestire le chiamate \gls{http} richiedenti dati relativi all'entità Account. Utilizzando la classe Athesys.ADCrm.Model::Crm (che implementa il design pattern Factory) vengono creati oggetti sui quali si possono invocare metodi per interrogare le \glo{API} di uno specifico CRM.
I dati recuperati vengono incapsulati in una risposta \gls{http} e ritornati ad ADProject.

\paragraph{Metodi:}\hfill
\begin{itemize}
	\itemsep0em 
	\item 
		\begin{lstlisting}
		Public async Task<HttpResponseMessage> Get(organizationId : string)
		\end{lstlisting}
		Metodo invocato quando viene inviata una richiesta \gls{http} GET alla route "api/organization/{organizationId}/accounts", ritornado una risposta \gls{http} contenente la lista di tutti gli Account censiti nel CRM.\\
		\textbf{\small Argomenti:}
		\begin{enumerate}[leftmargin=*]
			\itemsep0em 
			\item \begin{lstlisting}
			organizationId : string 
			\end{lstlisting}
			Rapprensenta l'identificativo univoco legato all'utenza aziendale con cui accedere al servizio CRM
		\end{enumerate}
		
	\item 
		\begin{lstlisting}
		Public async Task<HttpResponseMessage> GetById(organizationId : string, accountId : string)
		\end{lstlisting}
		Metodo invocato quando viene inviata una richiesta \gls{http} GET alla route "api/organization/{organizationId}/accounts/{accountId}", ritornando una risposta \gls{http} contenente l'Account avente id = {accountId}.\\
		\textbf{\small Argomenti:}
		\begin{enumerate}[leftmargin=*]
			\itemsep0em 
			\item 
				\begin{lstlisting}
				organizationId : string 
				\end{lstlisting}
				Rappresenta l'identificativo univoco legato all'utenza aziendale con cui accedere al servizio CRM
			\item 
				\begin{lstlisting}
				accountId : string
				\end{lstlisting}
				Rappresenta l'identificativo univoco legato ad un Account del CRM, attraverso il quale si andrà a richiedere i dati voluti
		\end{enumerate}
\end{itemize}

\subsubsection{ProposalsController (class)}

\paragraph{Descrizione:}
Classe che serve per gestire le chiamate \gls{http} richiedenti dati relativi all'entità Proposal. Utilizzando la classe Athesys.ADCrm.Model::Crm (che implementa il design pattern Factory) vengono creati oggetti sui quali si possono invocare metodi per interrogare le \glo{API} di uno specifico CRM. I dati recuperati vengono incapsulati in una risposta \gls{http} e ritornati ad ADProject.

\paragraph{Metodi:}\hfill
\begin{itemize}
	\itemsep0em 
	\item 
	\begin{lstlisting}
	Public async Task<HttpResponseMessage> Get(organizationId : string)
	\end{lstlisting}
	Metodo invocato quando viene inviata una richiesta \gls{http} GET alla route "api/organization/{organizationId}/proposals", ritornando una risposta \gls{http} contenente tutte le offerte commerciali censite nel CRM.\\
	\textbf{\small Argomenti:}
	\begin{enumerate}[leftmargin=*]
		\itemsep0em 
		\item \begin{lstlisting}
		organizationId : string 
		\end{lstlisting}
		Rappresenta l'identificativo univoco legato all'utenza aziendale con cui accedere al servizio CRM.
	\end{enumerate}
	
	\item 
	\begin{lstlisting}
	Public async Task<HttpResponseMessage> GetById(organizationId : string,  proposalId : string)
	\end{lstlisting}
	Metodo invocato quando viene inviata una richiesta \gls{http} GET alla route "api/organization/{organizationId}/proposals/{proposalId}", ritornando una risposta \gls{http} contenente l'offerta commerciale avente id = {proposalId}.\\
	\textbf{\small Argomenti:}
	\begin{enumerate}[leftmargin=*]
		\itemsep0em 
		\item 
		\begin{lstlisting}
		organizationId : string 
		\end{lstlisting}
		Rapprensenta l'identificativo univoco legato all'utenza aziendale con cui accedere al servizio CRM;
		\item 
		\begin{lstlisting}
		proposalId : string
		\end{lstlisting}
		Rappresenta l'identificativo univoco legato ad un offerta commerciale del CRM, attraverso il quale si andrà a richiedere i dati voluti.
	\end{enumerate}

	\item 
	\begin{lstlisting}
	Public async Task<HttpResponseMessage> GetByAccountId(organizationId : string,  accountId : string)
	\end{lstlisting}
	Metodo invocato quando viene inviata una richiesta \gls{http} GET alla route "api/organization/{organizationId}/accounts/{accountId}/proposals", ritornando una risposta \gls{http} contenente la lista delle proposte legate all'account avente id = {proposalId}.\\
	\textbf{\small Argomenti:}
	\begin{enumerate}[leftmargin=*]
		\itemsep0em 
		\item 
		\begin{lstlisting}
		organizationId : string 
		\end{lstlisting}
		Rapprensenta l'identificativo univoco legato all'utenza aziendale con cui accedere al servizio CRM;
		\item 
		\begin{lstlisting}
		accountId : string
		\end{lstlisting}
		Rappresenta l'identificativo univoco legato all'account di un azienda cliente CRM, attraverso il quale si andrà a richiedere i dati voluti.
	\end{enumerate}
\end{itemize}

\vfill
\subsection{Athesys.ADCrm.Common}

	\begin{figure}[H]
		\centering
		\rotatebox{90}{\includegraphics[width=.89\textheight,height=1.08\linewidth]{images/modules/common}}
		
		\caption{Diagramma UML del package Athesys.ADCrm.Common}
		\label{fig:common}
	\end{figure}


\newpage
\paragraph{Descrizione:}
Questo \glo{package} contiene il \glo{package} \textbf{Data} (avente le classi \gls{DTO}) e tutte le interfacce che dovranno essere implementate diversamente per ogni CRM che si vuole collegare all'applicazione. 
%TODO: da modificare la descrizione

\subsubsection{IIdentityProvider (interface)}

\paragraph{Descrizione:}
Interfaccia che dovrà essere implementata dalle classi concrete Athesys.ADCrm.Salesforce::IdentityProvider e\\ Athesys.ADCrm.Dynamics::IdentityProvider, essa definisce i metodi utilizzati per ottenere i \glo{token} di autenticazione e di refresh dai CRM, ed inoltre espone il metodo per effettuare inviare le \glo{query} di recupero dei dati.\\
La creazione di quest'interfaccia e l'implementazione dei metodi della stessa, sono stati sviluppati da un programmatore senior dell'azienda, come descritto nel paragrafo \ref{noteImplementazione}, Note di implementazione.
\paragraph{Metodi:}\hfill
\begin{itemize}
	\itemsep0em 	
	\item 
	\begin{lstlisting}
	 Task<AuthToken> GetAuthTokenAsync(string authorizationCode, string clientId, string clientSecret, string redirectUrl)
	\end{lstlisting}
	Metodo da implementare per ottenere i \glo{token} da aggiungere all'\glo{header} delle richieste \gls{http} per poter effettuare le query al CRM e il \glo{token} di refresh per otterene nuovi \glo{token} una volta scaduti. Tutti i paramentri passati al metodo sono essenziali per effettuare la suddetta richiesta.\\
	\textbf{\small Argomenti:}
	\begin{enumerate}[leftmargin=*]
		\itemsep0em 
		\item 
		\begin{lstlisting}
		authorizationCode : string
		\end{lstlisting}
		E' un codice che viene generato nel caso l'utente abbia inserito le credenziali d'accesso al CRM;
		%TODO: da aggiungere i riferimenti alla sezione;
		\item 
		\begin{lstlisting}
		clientSecret : string
		\end{lstlisting}
		E' un codice alfanumerico identificativo dell'applicazione, che non cambia nel tempo, generato dall'CRM. Esso è fondamentale per permettere il collegamento OAuth all'applicazione; 
		%TODO: in caso da modificare questa parte
		\item 
		\begin{lstlisting}
		clientId : string
		\end{lstlisting}
		E' un codice alfanumerico che non cambia nel tempo, generato dall'CRM. Esso è fondamentale per permettere il collegamento Oauth all'applicazione. 
		%TODO: in caso da modificare questa parte
		\item 
		\begin{lstlisting}
		redirectUrl : string
		\end{lstlisting}
		E' l'\glo{URL} su cui si vuole far arrivare le risposte del CRM.
		%TODO: in caso da modificare questa parte
	\end{enumerate}
	
	\item 
	\begin{lstlisting}
	Task<string> RenewAuthTokenAsync(string refreshToken, string clientId, string clientSecret)
	\end{lstlisting}
	Metodo da implementare per ottenere un nuovo \glo{token} di autorizzazione nel caso il precedente sia scaduto.\\
	\textbf{\small Argomenti:}
	\begin{enumerate}[leftmargin=*]
		\itemsep0em
		\item 
		\begin{lstlisting}
		refreshToken : string
		\end{lstlisting}
		E' un codice generato dall'CRM fornito una volta effettuato l'accesso con OAuth. Questo permette di ottenere nuovi \glo{token} di autorizzazione senza dover effettuare nuovamente la procedura di login;
		\item 
		\begin{lstlisting}
		clientId : string
		\end{lstlisting}
		E' un codice alfanumerico che non cambia nel tempo, generato dall'CRM. Esso è fondamentale per permettere il collegamento OAuth all'applicazione. 
		%TODO: in caso da modificare questa parte
		\item 
		\begin{lstlisting}
		clientSecret : string
		\end{lstlisting}
		E' un codice alfanumerico identificativo dell'applicazione, che non cambia nel tempo, generato dall'CRM. Esso è fondamentale per permettere il collegamento OAuth all'applicazione; 
		%TODO: in caso da modificare questa parte
	\end{enumerate}

	\item 
	\begin{lstlisting}
	 public async Task<QueryResponse> SendQueryWithRefresh(string resourceProvider,string accessToken,string refreshToken,string clientId, string clientSecret, Action<HttpRequestMessage> addQuryAction)
	\end{lstlisting}
	Metodo da implementare per inviare la richiesta \gls{http} con la \glo{query} per ottenere i dati desiderati dal CRM. Nel caso il \glo{token} d'autorizzazione passato fosse scaduto il metodo si occupa di recuperarne uno nuovo tramite la procedura di refresh. Nella successiva descrizione degli argomenti, per evitare inutili ripetizioni, vengono esclusi quelli già descritti nei sopracitati metodi.\\
	\textbf{\small Argomenti:}
	\begin{enumerate}[leftmargin=*]
		\itemsep0em
		\item 
		\begin{lstlisting}
		resourceProvider : string
		\end{lstlisting}
		E' l'\glo{URI} a cui si devono effettuare le \glo{query};
		\item 
		\begin{lstlisting}
		addQuryAction : Action<HttpRequestMessage>
		\end{lstlisting}
		E' il corpo della \glo{query} in formato di HttpRequestMessage. 
		%TODO: in caso da modificare questa parte
	\end{enumerate}
\end{itemize}

\subsubsection{IProposal (interface)}

\paragraph{Descrizione:}
Interfaccia che dovrà essere implementata dalle classi concrete Athesys.ADCrm.Salesforce::Proposal e\\ Athesys.ADCrm.Dynamics::Proposal, essa definisce i metodi utilizzati per interrogare i rispettivi CRM attraverso le \glo{API} REST esposte, restituendo la risposta \gls{http} incapsulata all'interno di un \gls{DTO} di tipo ResponseContainerData.

\paragraph{Metodi:}\hfill
\begin{itemize}
	\itemsep0em 
	\item 
	\begin{lstlisting}
	Task<ResponseContainerData<ProposalData>> GetAll()
	\end{lstlisting}
	Metodo da implementare per interrogare il CRM ritornando la lista di tutte le offerte commerciali censite nello stesso. Il metodo restituisce il \gls{DTO} ResponseContainerData istanziato al tipo ProposalData essendo una classe \glo{generic}.\\
	
	\item 
	\begin{lstlisting}
	Task<ResponseContainerData<ProposalData>> GetById(proposalId : string)
	\end{lstlisting}
	Metodo da implementare per interrogare il CRM ritornando l'offerta commerciale avente id = {proposalId}. Il metodo restituisce il \gls{DTO} ResponseContainerData istanziato al tipo ProposalData essendo una classe \glo{generic}.\\
	\textbf{\small Argomenti:}
	\begin{enumerate}[leftmargin=*]
		\itemsep0em 
		\item 
		\begin{lstlisting}
		proposalId : string
		\end{lstlisting}
		Rappresenta l'identificativo univoco legato ad un offerta commerciale del CRM, attraverso il quale si andrà a richiedere i dati voluti.
	\end{enumerate}
	
	\item 
	\begin{lstlisting}
	Task<ResponseContainerData<ProposalData>> GetByAccountId(accountId : string)
	\end{lstlisting}
	Metodo da implementare per interrogare il CRM ritornando la lista di tutte le offerte commerciali legate all'account avente id = {accountId}. Il metodo restituisce il \gls{DTO} ResponseContainerData istanziato al tipo ProposalData essendo una classe \glo{generic}.\\
	\textbf{\small Argomenti:}
	\begin{enumerate}[leftmargin=*]
		\itemsep0em
		\item 
		\begin{lstlisting}
		accountId : string
		\end{lstlisting}
		Rappresenta l'identificativo univoco legato all'account di un azienda cliente CRM, attraverso il quale si andrà a richiedere i dati voluti.
	\end{enumerate}
\end{itemize}
\vfill
\subsection{Athesys.ADCrm.API.Data}
\begin{figure}[H]
	\centering
	\includegraphics[width=\linewidth]{images/modules/Data}
	\caption{Diagramma UML del package Athesys.ADCrm.Common::Data}
	\label{fig:data}
\end{figure}

\paragraph{Description:}
Questo \glo{package} contiene tutti i vari \gls{DTO} definiti per ogni entità di dati che si vuole restituire alle chiamate effettuate da ADProject ed inoltre quelle necessarie per gestire le risposte le risposte provenienti dai CRM. 
Segue in questa sezione una breve descrizione delle classi principali, sorvolando sui metodi delle stesse in quanto sono tutti \glopl{accessor} \glo{getter} e \glo{setter}.


\subsubsection{ProposalData (class)}
\paragraph{Descrizione:}
Classe che rappresenta un oggetto \gls{DTO} per modellare l'entità \textit{Proposal}.
In essa sono presenti i metodi \glo{getter} e \glo{setter} e tutti i campi dati (privati) contenenti gli elementi che caratterizzano un offerta commerciale, quali: l'identificativo univoco della stessa, il cliente a cui è stata proposta, il costo totale, un eventuale sconto e molti altri dati.
Gli oggetti di questa classe, contenente i dati provenienti dal CRM, verranno incapsulati all'interno della risposta \gls{http} restituita ad ADProject.

\subsubsection{CrmResponseData (class)} \label{crmResponseDataClass}
\paragraph{Descrizione:}
Classe che rappresenta un oggetto \gls{DTO} per modellare le risposte \gls{http} provenienti dal \gls{CRM}.
Questa classe ha solamente due campi dati (ed i relativi \glopl{accessor}):
\begin{itemize}
	\item 	
	\begin{lstlisting}
	Private HttpStatusCode Status
	\end{lstlisting}
	In questo campo dati viene salvato il codice di stato \gls{http} della risposta mandata dal \gls{CRM};
	\item
	\begin{lstlisting}
	Private string Content
	\end{lstlisting}
	In questo campo dati viene salvato il \glo{JSON} della risposta mandata dal \gls{CRM}.
\end{itemize}

\subsubsection{ResponseContainerData (class)} \label{responseContainerDataClass}
\paragraph{Descrizione:}
Classe che rappresenta un oggetto \gls{DTO} per modellare le risposte \gls{http} da inviare a ADProject.
Questa classe ha tre campi dati (ed i relativi \glopl{accessor}):
\begin{itemize}
	\item 	
	\begin{lstlisting}
	private List<T> Payload
	\end{lstlisting}
	In questo campo dati è una lista di \glopl{generic} che viene istanziata al tipo di uno degli oggetti \gls{DTO} di questo \glo{package} (AccountData,ProposalData,ProductData,ContactData,UserData,ProductCategoryData o OrganizationData);
	
	\item 	
	\begin{lstlisting}
	private HttpStatusCode Status
	\end{lstlisting}
	In questo campo dati viene salvato il codice di stato \gls{http} che dovrà assumere la risposta da inviare ad ADProject;
	
	\item
	\begin{lstlisting}
	private string Description
	\end{lstlisting}
	In questo campo dati, in caso di errore, viene aggiunta una breve descrizione testuale dello stesso.
\end{itemize}




\subsection{Athesys.ADCrm.Model}
Questo \glo{package} contiene tutte le classi concrete in comune tra i \gls{CRM} che si vogliono collegare all'applicazione.
La classe principale del \glo{package} è Athesys.ADCrm.Model::Crm.
\subsubsection{Crm (class)} \label{crmClass}
\begin{figure}[H]
	\centering
	\includegraphics[width=\linewidth]{images/factoryInADCrm}
	\caption{Diagramma semplificato del pattern Factory per ADCrm}
	\label{fig:factoryInADCrm}
\end{figure}
\paragraph{Descrizione:}
Questa classe implementa il design pattern Factory ed è usata dai controllers per produrre oggetti concreti di un particolare \gls{CRM} (su cui invocare i metodi per recuperare i dati dallo stesso) senza conoscere a priori il tipo di \gls{CRM} richiesto (Dynamics o SalesForce).


Di seguito vengono riportati alcuni metodi significativi della classe per esemplificarne il funzionamento generale.


\paragraph{Metodi:}\hfill
\begin{itemize}
	\itemsep0em 
	\item 
	\begin{lstlisting}
	  public static async Task<OrganizationData> BuildOrganization(string organizationId)	
	\end{lstlisting}
	Metodo che costruisce un oggetto di tipo OrganizationData, appartenente ad un particolare \gls{CRM}, in base alla tipologia del parametro passato.
	Tutti i campi dati dell'oggetto ritornato vengono impostati dal database dell'applicazione che contiene i parametri di connessione ai vari \gls{CRM}.\\
	Questo metodo è stato sviluppato da un programmatore senior dell'azienda, come descritto nel paragrafo \ref{noteImplementazione}, Note di implementazione.\\
	\textbf{\small Argomenti:}
	\begin{enumerate}[leftmargin=*]
		\itemsep0em 
		\item 
		\begin{lstlisting}
		organizationId : string
		\end{lstlisting}
		Rappresenta la tipologia di \gls{CRM} a cui ci si deve collegare. Questo dati si ottiene grazie alle route definite nella sezione \ref{apiRest} (eg. api/organization/\{\textbf{organizationId}\}/accounts).
	\end{enumerate}
	
	\item 
	\begin{lstlisting}
     public static IProposal BuildProposal(OrganizationData org)
	\end{lstlisting}
	Metodo che, in base alla tipologia di organizzazione passata per parametro, costruisce un oggetto Proposal per un particolare tipo di \gls{CRM}\\
	\textbf{\small Argomenti:}
	\begin{enumerate}[leftmargin=*]
		\itemsep0em
		\item 
		\begin{lstlisting}
		org : OrganizationData
		\end{lstlisting}
		Rappresenta la tipologia di \gls{CRM} da cui si vogliono recuperare le proposte commerciali.
	\end{enumerate}
\end{itemize}
 
\subsection{Athesys.ADCrm.Salesforce}\label{salesforce}
Questo \glo{package} contiene tutte le classi che implementano le interfacce definite in Athesys.ADCrm.Common e, restituiscono ai controllers di Athesys.ADCrm.API i vari oggetti \gls{DTO} delle entità del \glo{package} Athesys.ADCrm.Common.Data.
Di seguito vengono descritte brevemente le uniche tre classi che hanno un comportamento diverso da quanto sopracitato.

\subsubsection{ResponceMapper (class)}\label{responceMapperClass}
\paragraph{Descrizione:}
Questa classe ha la funzione di mappare le risposte provenienti dal \gls{CRM} modificandole in base alle esigenze di ADProject.
Per esempio se dal \gls{CRM} arrivasse una risposta d'errore con codice di stato \gls{http} 500 (quindi un errore interno del server), non si dovrà restituire ad ADProject una risposta lo stesso codice di stato, in quanto verrebbe interpretato come un errore del servizio e non del \gls{CRM}. C'è quindi l'esigenza di modificare ad-hoc le risposte per renderle fruibili dal richiedente delle stesse.
%TODO: controllare questa parte

\subsection{Athesys.ADCrm.Dynamics}\label{dynamics}
Questo \glo{package} contiene tutte le classi che implementano le interfacce definite in Athesys.ADCrm.Common e restituiscono ai controller di Athesys.ADCrm.API i vari oggetti \gls{DTO} di tipo Athesys.ADCrm.Common.Data al fine di incapsularli nella risposta \gls{http}.
Viene omessa la descrizione delle classi e dei metodi in quanto risulta identica a quella del \glo{package} Athesys.ADCrm.Salesforce.

\subsection{Database}
Il database è utilizzato per salvare i dati necessari a collegarsi per effettuare l'accesso e le richieste ai \gls{CRM}, vista la bassissima mole di dati da salvare si è deciso di optare per un semplicissimo database \glo{SQL} composto da una singola tabella, di nome Organization, avente la seguente struttura:
\begin{figure}[H]
	\centering
	\includegraphics[width=0.7\linewidth]{images/schemaDB}
	\caption{Struttura tabella Organization del database}
	\label{fig:schemadb}
\end{figure}

\begin{itemize}
	\item \textbf{Id} : è un identificativo univoco alfanumerico, marcato come \textit{primary key} per poter distinguere i dati di  un'applicazione \gls{CRM} di una specifica azienda
	\item \textbf{Name} : è il nome del software \gls{CRM} (eg. SalesForce o Dynamics)
	\item \textbf{Type} : è un campo intero per distinguere la tipologia di \gls{CRM} (eg. \textit{SalesForce} o \textit{Dynamics})  
	\item \textbf{ResoureProvider} :
	Ogni software \gls{CRM} \textit{cloud-based} viene ospitato su un server, e per identificare a quale di questi ci si dovrà collegare, bisogna inserire questo campo nel \glo{URL} (eg. \url{https://emea.salesforce.com} dove il campo ResoureProvider corrisponde alla voce \textit{"emea"} )
	%TODO: verificare il termine hostare
	\item \textbf{ClientId} : è l’identificativo univoco (pubblico) del client ed è uno dei dati fondamentali che le applicazioni \gls{CRM} devono fornire per permettere il collegamento di terze applicazioni-web tramite procedura OAuth
	\item \textbf{ClietSecret} :  è l’identificativo segreto del client ed è il secondo dato fondamentale che le applicazioni \gls{CRM} devono fornire per permettere il collegamento di terze applicazioni-web tramite procedura OAuth
	\item \textbf{AuthToken} : è il campo contenente la risposta \glo{JSON} restituita da un \gls{CRM} dopo la procedura di autenticazione OAuth; quindi non solo conterrà il \glo{token} da allegare per poter fare le richieste, ma anche il \glo{token} di refresh ed altri dati. Ne segue un esempio.
	
	
	\begin{lstlisting}[language=json,firstnumber=1]
	{
	"id":"https://login.salesforce.com/***",
	"issued_at":"1278448101416",
	"refresh_token":"***",
	"instance_url":"https://***.salesforce.com/",
	"signature":"***",
	"access_token":"***"
	}
	\end{lstlisting}
	
\end{itemize}

Per realizzare questo database \glo{SQL} è necessario avere le credenziali aziendali d'accesso a \textit{Microsoft SQL Server}, un \textit{\glo{DBMS}} prodotto da \textit{Microsoft}. Questo sistema contiene svariati database aziendali con dati riservati, quindi la realizzazione della base di dati per l'applicazione \textbf{ADCrm} e il collegamento con lo stessa è stato realizzato da un programmatore senior dell'azienda, come descritto nel paragrafo \ref{noteImplementazione} Note di implementazione. 

\subsection{Diagramma di sequenza}
	\begin{figure}[H]
	\centering
	\rotatebox{90}{\includegraphics[width=.87\textheight,height=1.08\linewidth]{images/sdProposals}}
	
	\caption{Diagramma di sequenza per recuperare le entità Account da un \gls{CRM} }
	\label{fig:sdProposals}
\end{figure}

L'immagine soprastante è il diagramma di sequenza utile per facilitare la comprensione della catena di chiamate che viene effettuata per recuperare tutti i dati dei clienti dell'azienda dall'applicazione \gls{CRM}.

\begin{itemize}
	\item \textbf{AccountsController:} è l'oggetto che si occupa di invocare il metodo appropriato ogni volta che viene instradata una chiamata all'\glo{endpoint} "\textit{accounts}" (sez. \ref{accountsController})  
	\item \textbf{Crm:} è l'oggetto che implementa il design pattern \textit{Factory} (sez. \ref{factory}) costruendo oggetti per recuperare i dati da Salesforce o Dynamics (sez. \ref{crmClass})
	\item \textbf{Account:} è l'oggetto attraverso il quale vengono recuperati tutti i dati richiesti, riguardanti le aziende clienti, dal \gls{CRM}
	\item \textbf{CrmResponseData:} è un oggetto \gls{DTO} nel quale viene incapsulata la risposta \glo{JSON} restituita dal \gls{CRM} (sez. \ref{crmResponseDataClass})
	\item \textbf{AccountData:} è un oggetto \gls{DTO} che rappresenta l'account di un azienda cliente
	\item \textbf{ResponseContainerData:}
	è un oggetto \gls{DTO} nel quale viene incapsulata la risposta \glo{JSON} che si restituirà all'applicazione chiamante, ADProject (sez. \ref{responseContainerDataClass})
	\item \textbf{ResponseMapper:} è la classe di utilità che si occupa di rimappare i codici \gls{http} delle risposte provenienti dal \gls{CRM} e di, in caso di errore modificare il JSON di risposta (sez. \ref{responceMapperClass})

\end{itemize}


\chapter{Verifica e validazione}\label{verifica_validazione}
\section{Verifica}
L'obbiettivo della verifica è quello di accertare che l'esecuzione di un dato processo o attività non abbia introdotto errori nel sistema.
Per il processo di verifica mediamente sono usate due tecniche principali:
\begin{itemize}
	\item \textbf{Analisi statica:} tipologia di analisi che mira a verificare il codice senza doverlo eseguire
	\item \textbf{Analisi dinamica:} tipologia di analisi che mira a verificare il codice attraverso l'esecuzione dello stesso
\end{itemize}

\subsection{Software Testing}
I test sono stati effettuati attraverso gli strumenti integrati in Visual Studio (facendo anche largo uso del \textit{\glo{Debugger}}), permettendo di rilevare e correggere piccoli errori mediamente dovuti a distrazione ed all'inesperienza nell'utilizzo dei linguaggi coinvolti.\\
I test di integrazione e di sistema sono stati condotti a mano mentre si è fatto uso di test strutturali per verificare il corretto comportamento delle unità di software.

\paragraph{Test strutturale}
o \textit{white box testing}, sono utilizzati per verificare la logica interna del codice di un unità software. Questo obbiettivo viene ottenuto fornendo in input una serie di dati che permetta di percorrere tutti i cammini (sequenze di istruzioni attraversate durante un'esecuzione) all'interno del modulo testato, controllando che esso produca output attesi .\\
Dato che questa tipologia di test è dipendente dai dati forniti in input e ottenuti in output, nello svolgimento progetto di stage è stato fatto largo uso di \textbf{Stub} e \textbf{Driver}.

\begin{figure}[H]
	\centering
	\includegraphics[width=\linewidth]{images/stubDriver}
	\caption{Stub e Driver}
	\label{fig:stubdriver}
\end{figure}

\paragraph{Stub} 
in informatica i "test stubs" sono programmi o porzioni di codice che simulano il comportamento di componenti software o moduli (imitando l'output di un unità chiamata), essi vengono spesso utilizzati per sostituire elementi non ancora codificati.\\
Questa metodologia è stata utilizzata per testare il funzionamento complessivo del sistema prima di implementare i metodi (definiti nelle interfacce di Athesys.ADCrm.Common) delle classi contenute nei \glo{package} Athesys.ADCrm.SalesForce (sez. \ref{salesforce}) e Athesys.ADCrm.Dynamics (sez. \ref{dynamics}).

\paragraph{Driver} 
i "test driver" sono porzioni di codice per simulare il comportamento di un unità chiamante, non ancora implementata, per verificare se l'unità chiamata funziona correttamente e produce output attesi.\\
Questa tipologia di test è stata utilizzata per simulare tutte le chiamate alle API di ADCrm, cosi da assicurarsi che il sistema fosse funzionante dopo ogni eventuale modifica a \glossary{query} o metodi.
\section{Validazione}
Il processo di validazione è indispensabile e serve per accertarsi, mediante misurazioni e prove oggettive, che il prodotto (in questo caso ADCrm) risponda a tutte le specifiche attese dall'utente.
Per accertarsi di ciò vengono effettuati i test di accettazione, dove il cliente finale controlla che il prodotto risponda correttamente a tutti i requisiti fissati nella fase iniziale di analisi.
Questi test sono stati effettuati in presenza del referente aziendale per il progetto di stage eseguendo un controllo completo di tutte le funzionalità del software.


\chapter{Conclusioni}\label{conclusioni}
\section{Copertura dei requisiti}
A conclusione del progetto di stage la quasi totalità degli obbiettivi prefissati sono stati raggiunti; i requisiti sono stati soddisfatti secondo le seguenti percentuali:
\begin{itemize}
	\item \textbf{Requisiti Obbligatori:} 100\%
	\item \textbf{Requisiti Opzionali:} 88\%
\end{itemize}
Non è stato possibile, per mancanza di tempo, concludere solamente il requisito opzionale \textbf{R-2F5.5} (riguardante il recupero dati dei prodotti su Microsoft Dynamics) in quanto, per poter ridurre al minimo il numero di chiamate http al CRM, è stato necessario modificare le \glo{query} per interrogare Dynamics.
\section{Sviluppi futuri}
Attraverso il progetto di stage è nato il servizio web \textbf{ADCrm}, ma non può ancora definirsi concluso. Sono infatti previsti una serie di incrementi che dovranno essere attuati nell'immediato futuro per rendere il software totalmente operativo:
\begin{itemize}
	\item conclusione della integrazione con Microsoft Dyamics e completamento del requisito \textbf{R-2F5.5};
	\item incrementare la parte di gestione degli errori attraverso un sistema completo di \glo{logging} delle eccezioni sul servizio ADCrm.
\end{itemize}
Inoltre l'azienda ha ipotizzato alcuni sviluppi futuri riguardo l'incremento delle funzionalità di ADProject nell'ambito dei CRM:
\begin{itemize}
	\item dare la possibilità di effettuare operazioni di scrittura da ADProject sul CRM;
	\item fornire opzioni di sincronizzazione da ADProject verso i CRM;
	\item integrare ulteriori tipologie di software CRM.
\end{itemize}

\section{Considerazioni personali}
Durante l'insegnamento del corso di Ingegneria del Software, è stata più volte sottolineata l'importanza dello stage come conclusione del percorso formativo di ogni studente di Informatica, ma solo durante lo stesso si riesce ad apprezzare realmente quanto questo sia essenziale.\\
Il primo contatto con Athesys s.r.l è avvenuto grazie a STAGE-IT ed è stata l'azienda che, oltre ad aver presentato progetti estremamente interessanti, mi ha messo a mio agio durante tutti gli incontri avvenuti. Avendo già avuto in precedenza un'esperienza lavorativa nell'ambito dell'informatica, ho imparato che l'ambiente ed il team di lavoro sono tanto importanti quanto il progetto che si andrà a svolgere, ed è stato il connubio di questi due elementi che, a seguito di incontri e colloqui, mi ha dato la certezza della scelta di Athesys s.r.l per lo stage.\\
All'inizio mi mancavano molte delle conoscenze necessarie per realizzare il progetto, ma questo l'ho vissuto come una sfida, superata grazie alle competenze acquisite in questi anni di studio. La realizzazione di ADCrm mi ha fatto capire che il percorso di laurea in informatica non mi ha insegnato solamente dei linguaggi di programmazione, bensì mi ha dato la forma mentis per apprenderli ed usarli tutti, per reagire a problemi inaspettati e trovare soluzioni per superarli.\\ 
Ritengo quindi questa esperienza estremamente positiva, sia per gli obbiettivi raggiunti sia per le persone che ho conosciuto e che mi hanno affiancato durante lo stage.


\cleardoublepage
%**************************************************************
% Glossary and bibliography
%**************************************************************
%\printglossaries
\printnoidxglossaries
\bibliographystyle{unsrt}
%\bibliography{thesis}


%**************************************************************
% Ringraziamenti
%**************************************************************

%\chapter*{Ringraziamenti}
%rr 

%\noindent\textit{\myLocation, \myTime}
%\hfill \myName
%\endgroup
\end{document}